% !TEX root = ../../analisis.tex
%%%%%%%%%%%%%%%%%%%%%%%%%%%%%%%%%%%%%%%%%%%%%%%%%%%%%%%%%%%%%%%%%%%%%%%%%
%%%%%%%%%%%%%%%%%%%%%%%%%%%%%%%%%%%%%%%%%%%%%%%%%%%%%%%%%%%%%%%%%%%%%%%%%
%%%%%%%%%%%%%%%%  ESTO UNICAMENTE SIRVE COMO REFERENCIA   %%%%%%%%%%%%%%%
%%%%%%%%%%%%%%%%%%%%%%%%%%%%%%%%%%%%%%%%%%%%%%%%%%%%%%%%%%%%%%%%%%%%%%%%%
%%% AQUI SE AGREGARÁN TODOS LOS ARCHIVOS PARA LOS CASOS DE USO DE UN PROCES, HABRÁ UNO D ESTOS ARCHIVOS POR PROCESO


\chapter{Modelo de comportamiento del módulo de Horarios}

En este capítulo se describen los casos de uso referentes al registro y modificación de la información de las escuelas y del comité asociado a cada una de ellas. 

\bigskip

     \begin{objetivos}[Elementos de un caso de uso]
	\item {\bf Resumen:} Descripción textual del caso de uso.
	\item {\bf Actores:} Lista de los 
	 que intervienen en el caso de uso.
	\item {\bf Propósito:} Una breve descripción del objetivo que busca el actor al ejecutar el caso de uso.
	\item {\bf Entradas:} Lista de los datos de entrada requeridos durante la ejecución del caso de uso.
	\item {\bf Salidas:} Lista de los datos de salida que presenta el sistema durante la ejecución del caso de uso.
	\item {\bf Precondiciones:} Descripción de las operaciones o condiciones que se deben cumplir previamente para que el caso de uso pueda ejecutarse correctamente.
	\item {\bf Postcondiciones:} Lista de los cambios que ocurrirán en el sistema después de la ejecución del caso de uso y de las consecuencias en el sistema.
	\item {\bf Reglas de negocio:} Lista de las reglas que describen, limitan o controlan algún aspecto del negocio del caso de uso.
	\item {\bf Errores:} Lista de los posibles errores que pueden surgir durante la ejecución del caso de uso.
	\item {\bf Trayectorias:} Secuencia de los pasos que ejecutará el caso de uso.
    \end{objetivos}


\cfinput{ModeloComportamiento/modulo-horarios/cuh1/uc}
\cfinput{ModeloComportamiento/modulo-horarios/cuh1.1/uc}
\cfinput{ModeloComportamiento/modulo-horarios/cuh1.2/uc}
\cfinput{ModeloComportamiento/modulo-horarios/cuh1.3/uc}
%\cfinput{ModeloComportamiento/modulo-horarios/cuh1.4/uc}
%\cfinput{ModeloComportamiento/modulo-horarios/cuh1.4.1/uc}


	
