% !TEX root = ../introduccion.tex
%SUPERFICIES FORESTALES

%----------------------------------------------------------
\section{Entorno de trabajo}

    El entorno de trabajo es el medio por el cual el usuario interactúa con el sistema para poder gestionar la información referente a la ESCOM. En este capítulo se describe el comportamiento y los elementos que conforman el entorno de trabajo de la ESCOMapp, como son: la disposición de los elementos principales y comunes de las pantallas, los colores, la iconografía, componentes, etc. \bigskip

%    \begin{objetivos}
%      \item Describir las áreas principales del entorno de trabajo.
%      \item Describir la iconografía utilizada en las pantallas.
%      \item Describir el mapa de navegación del sistema.
%      \item Describir los componentes principales de las pantallas, tales como: controles de entrada, datos obligatorios, separadores, tablas de resultados, entre otros.
%    \end{objetivos}
%\\\\\\\\\\\\\\\\
%----------------------------------------------------------

%\subsection{Diseño}
%
%  El diseño de las pantallas del sistema sigue un enfoque minimalista que permite a los usuarios trabajar sin gran dificultad y sin distracción. 
%  Las pantallas son consistentes, ya que tienen un diseño homogéneo y cuentan con componentes comunes; la consistencia facilita al usuario la interacción
%  con el sistema a medida que hace uso del mismo. En la figura~\ref{fig:entornoDeTrabajo} se muestran los elementos principales que conforman las pantallas del sistema, 
%  dichos elementos se describen a continuación:
%
%  \begin{figure}[ht!]
%      \begin{center}
%	  \fbox{\includegraphics[width=.8\textwidth]{images/layout}}
%	  \caption{Entorno de trabajo del sistema.}
%	  \label{fig:entornoDeTrabajo}
%      \end{center}
%  \end{figure}
%
%    \begin{enumerate}
%        \item {\bf Encabezado:} el encabezado tiene la finalidad de mostrar la imagen institucional de la dependencia a la cual pertenece, es decir, la imagen institucional del Gobierno del Estado de México.
%        \begin{itemize}
%            \item Ancho: $100\%$ del ancho de la ventana del navegador.
%            \item Alto: $90px$.
%        \end{itemize}
%
%        \item {\bf Menú horizontal:} muestra las opciones generales de navegación para los distintos tipos de usuarios.
%        \begin{itemize}
%            \item Ancho: $100\%$ del ancho de la pantalla del navegador.
%            \item Alto: $40px$.
%        \end{itemize}
%        
%        \item {\bf Menu vertical:} es el área destinada al menú vertical que contendrá los vínculos necesarios para ingresar a las opciones que proporcione el sistema a cada uno de los distintos perfiles de usuarios.
%        
%        El menú vertical no se encontrará visible para los perfiles de usuario que no requieran del mismo y este espacio será utilizado por el área de trabajo (ver siguiente punto).
%        
%        \begin{itemize}
%            \item Ancho: $20\%$ del ancho de la pantalla del navegador.
%            \item Alto: autoajustable al contenido.
%        \end{itemize}
%        
%        \item {\bf Área de trabajo:} en esta sección los usuarios visualizarán los elementos que el sistema proporciona para la realización de las tareas contempladas en el mismo. Aquí se desplegarán formularios para captura, tablas, imágenes, gráficas y demás elementos contenidos en el sistema.\\
%        
%        El contenido en esta sección se visualizará centrada con base en el ancho y alineado a la parte superior de la misma. Todas las pantallas deberán contar con un título alineado al centro del área de trabajo. 
%        \begin{itemize}
%            \item Ancho: ancho mínimo $500px$, $65\%$ del ancho de la ventana del navegador web cuando el menu vertical esta visible o el $80\%$ del ancho de la ventana del navegador web en ausencia del menu vertical.
%            \item Alto: autoajustable al contenido con un mínimo de $400px$.
%        \end{itemize}
%        
%        \item {\bf Pie:} esta sección contendrá la información de contacto de la unidad correspondiente de la Secretaría del Medio Ambiente del Gobierno del Estado de México.
%        \begin{itemize}
%            \item Ancho: $80\%$ del ancho de la ventana del navegador.
%            \item Alto: $84px$
%        \end{itemize}
%        
%        \item {\bf Información legal:} muestra una leyenda con información legal referente a la propiedad y uso del sistema.
%        \begin{itemize}
%            \item Ancho: $100\%$ del ancho de la ventana del navegador.
%            \item Alto: 24px.
%        \end{itemize}
%        
%        \item {\bf Información de sesión:} esta sección será visible sólo cuando un usuario ingrese al sistema. En ella se mostrarán las opciones para cambiar la contraseña de acceso al mismo, el nombre de usuario y el cierre de sesión.
%        \begin{itemize}
%            \item Ancho: ajustable al contenido.
%            \item Alto: $30px$;
%        \end{itemize}
%    \end{enumerate}

  
%\subsection{Pantalla de bienvenida}
%\label{ch:Interaccion:PantallaBienvenida}


%----------------------------------------------------------

%\subsection{Componentes utilizados}

 % \subsubsection{Pantalla emergente}
  
  
%----------------------------------------------------------
%\subsection{Datos de sesión}
%\label{ch:Interaccion:DatosSesion}


%\subsection{Iconografía}

%  En las pantallas se utilizan diversos íconos para denotar las operaciones que el actor puede realizar sobre el sistema. Los íconos se diseñaron con base en los perfiles de actor y en la operación que podrán realizar 
%  después del evento {\it clic} sobre ellos.  A continuación se describe la la funcionalidad de cada uno de ellos:\\\\

%  \begin{UClist}
    %\UCli \botKo: Permite eliminar un registro del sistema, por ejemplo: un integrante de línea de acción, una actividad en el plan de acción, etc. Al utilizar este ícono no se podrá recuperar la información eliminada. 
%    \UCli \botOk: Se utiliza para aprobar la solicitud de inscripción de una escuela al programa.
%     \UCli \botReg Se utiliza para solicitar el registro de información en el sistema, por ejemplo: un nuevo integrante de línea de acción, información correspondiente al diagnoóstico, etc. 
  %    \end{UClist}


%----------------------------------------------------------
  
\subsection{Organización}

Las funcionalidades del sistema se encuentra organizadas por menús. Cada actor accede a un menú diferente dependiendo de su perfil, ya que este describe el ciclo de trabajo y las funciones que el actor puede realizar.


\hypertarget{menu:SecretariaAdministracion}{}	
\subsection{Menú Principal: Secretaría de Administración}
En la figura \ref{MN1} se muestra el menú mediante el cual la Secretaría de Administración puede acceder a sus funcionalidades.
\IUfig[.7]{menus/SA.png}{MN1}{Menú de la Secretaría de Administración}
%------------------------------------------------------------------------------------------

\hypertarget{menu:CoordinacionControlEscolar}{}	
\subsection{Menú Principal: Coordinación de Control Escolar}
En la figura \ref{MN2} se muestra el menú mediante el cual la Secretaría de Administración puede acceder a sus funcionalidades.
\IUfig[.7]{menus/CO.png}{MN2}{Menú de la Coordinación de Control Escolar}
