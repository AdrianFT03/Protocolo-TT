%========ANÁLISIS DEL SENSOR DE TEMPERATURA=======

\section{Análisis de componentes del sistema}
Para determinar los componentes del sistema, se llevó a cabo la investigación y comparación de diferentes sensores para realizar las mediciones de temperatura y pulso cardíaco, así como de diferentes módulos de comunicación y microcontroladores. Se consideraron diferentes características de interés de cada componente para seleccionar el más conveniente para el sistema.


\subsection{Sensor de temperatura}
Para determinar el sensor de temperatura, se realizó una tabla comparativa considerando diversos sensores de tipo \textit{grado clínico}, que por las características que presentan son adecuados para su uso en aplicaciones médicas. \\

Las principales características a considerar para la selección del sensor fue el rango de temperatura de medición, que para el sistema se requiere el rango de temperatura corporal humana, y la precisión de las mediciones, dado que se requiere de gran precisión puesto que pequeñas variaciones podrían significar un cambio en el estado fisiológico del paciente.\\

Adicionalmente, se consideraron el tipo de sensor, analógico o digital, y la interfaz de comunicación que proporcionan.\\

En la Tabla \ref{analisis:sensorTemperatura} se muestran los datos de los diferentes sensores para cada característica. \\


	%\begin{sidewaystable}
		\begin{table}[htbp!]
			\begin{center}
			\scalebox{0.85}[0.95]{
			\begin{tabular}{|c|c|c|c|c|c|c|c|c|}
				\hline
				\thead{Modelo}&\thead{Rango \\ ($^{\circ}$C)}&\thead{Tipo}&\thead{Interfaz}&\thead{Precisión \\ ($^{\circ}$C)}&\thead{Resolución \\ (bits)}&\thead{Voltaje \\ (V)}&\thead{Corriente \\ (A)}&\thead{Precio \\ (USD)}\\
				\hline
				\hline
				MAX30205 & 0 a 50 & Digital& I2C& $\pm0.1$ &16 & 2.7 - 3.3&$600\mu$&1.60 \\
				\hline
				TMP101 & -55 a 125 & Digital& I2C, SMBus& $\pm1$ &9-12 & 2.7 - 5.5&$45\mu$&1.85 \\
				\hline
				LM73 & -40 a 150 & Digital& I2C, SMBus& $\pm1$ &11-14 & 2.7 - 5.5&$550\mu$&1.78 \\
				\hline
				TSYS01 & -40 a 125 & Digital& I2C, SPI& $\pm0.1$ &16 & 3.2 - 3.6&$< 12.5\mu$&14.95 \\
				\hline
				LM73 & -40 a 150 & Digital& I2C& $\pm0.1$ &14 & 1.9 - 3.6&$120\mu$&3.09 \\
				\hline
			\end{tabular}}
			\caption{Comparativa de sensores de temperatura.}
			\label{analisis:sensorTemperatura}
			\end{center}
		\end{table}
%	\end{sidewaystable}
\pagebreak

Se decidió seleccionar el sensor \textbf{MAX30205} mostrado en la figura \ref{fig:AnalisisMax30205} debido a que el rango de medición de temperatura es el más cercano al rango de temperatura corporal, que su error es de $\pm0.1^{\circ}$C cuando realiza mediciones entre los 37$^{\circ}$C y los 39$^{\circ}$C y que tiene un bajo costo. \\

Este sensor medirá la temperatura, convertirá los datos en formato digital y mediante la interfaz de comunicación I2C transmitirá los resultados de la conversión al microcontrolador cada que éste lo solicite.\\

		\begin{figure}[htbp!]
			\centering
			\fbox{\includegraphics[width=0.6\textwidth]{Analisis/imagenes/MAX30205.jpg}}
			\caption{MAX30205}
			\label{fig:AnalisisMax30205}
		\end{figure}
	\clearpage
