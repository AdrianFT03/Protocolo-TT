\section{Aplicación móvil}
Una aplicación móvil es un programa informático diseñado para funcionar en dispositivos móviles que permite que el usuario lleve a cabo una o varias operaciones. Una aplicación móvil puede ser limitada o amplia, sencilla o compleja y tanto en un caso como en el otro, está perfectamente controlada por la persona u organización que la haya diseñado. \cite{gardnerApp} \\

\subsection{Tipos de aplicaciones móviles}
\begin{itemize}
	\item \textbf{Aplicaciones nativas:} Una aplicación nativa es una aplicación desarrollada con herramientas específicas para que éstas se ejecuten en el sistema operativo, llamado Software Development Kit o SDK. Estas aplicaciones pueden acceder a los Sistemas Operativos del equipo móvil para facilitar el acceso a todas las características del hardware del móvil como brújula, cámara, correo, GPS, etc. \\ 
	
	Como este tipo de aplicaciones se instalan en el dispositivo, no es necesario que estén  conectadas a Internet, pero demandarán una tienda de aplicaciones desde donde realizan el proceso de descarga e instalación. \\
	
	\item \textbf{Aplicaciones web:} Una aplicación web es una aplicación tipo web desarrollada con HMTL5, CSS3, JQuery Mobile y JavaScript para su uso en dispositivos móviles como SmartPhones o Tabletas. Son aplicaciones que pueden ser ejecutadas en múltiples plataformas ya que no hacen uso del sistema operativo del equipo, sino del navegador del mismo para su ejecución. Esto significa que no se instalan en el dispositivo y consiguen una experiencia de operación muy similar al nativo, pero requieren conexión constante a Internet. \\
	
	Una de las principales ventajas de una aplicación Web es su soporte para múltiples plataformas y el bajo costo de desarrollo además de no necesitar tienda de aplicaciones para su distribución. \cite{ibmApp} \cite{demetrio2013} \cite{mobileApps2014}
	
\end{itemize}