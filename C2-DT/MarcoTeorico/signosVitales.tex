\section{Sensores}
%	Los signos vitales son indicadores que reflejan el estado fisiológico de los órganos vitales (cerebro, corazón, pulmones). Sus variaciones expresan cambios que ocurren en el organismo, algunos de índole fisiológico y otros de tipo patológico. Los valores considerados normales se ubican dentro de rangos y estos rangos varían según la edad y en algunos casos también con el sexo. \cite{aguayoChile} \cite{cobo2011} Los cuatros principales signos vitales son: 
%	\begin{enumerate}
%		\item Frecuencia cardíaca.
%		\item Frecuencia respiratoria.
%		\item Presión arterial.
%		\item Temperatura.
%	\end{enumerate}
%	
%	\subsection{Frecuencia cardíaca}
%	El pulso está representado por la expansión rítmica de las arterias producida por el pasaje de sangre que es bombeada por el corazón originada en la contracción del ventrículo izquierdo, y que resulta en la expansión y contracción regular del calibre de las arterias; representa el rendimiento del latido cardíaco y la adaptación de las arterias. \cite{signos2017} \cite{aguayoChile} \cite{valoresUNAM}\\
%	
%	Los valores del Pulso arterial se miden a partir de la “Frecuencia Cardíaca” o sea el número de pulsaciones o latidos que ocurren en “Un Minuto”. La frecuencia cardíaca varía dependiendo de diferentes factores, como: la edad, sexo, actividad física, estado emocional, fiebre, medicamentos y hemorragias.
%	
%	\begin{table}[htbp]
%		\begin{center}
%			\begin{tabular}{|l|l|}
%				\hline
%				\textbf{Edad} & \textbf{Pulsaciones por minuto} \\
%				\hline \hline
%				Recién nacido & 120 - 170  \\
%				\hline
%				Lactante menor & 120 - 160  \\
%				\hline
%				Lactante mayor & 110 - 130  \\
%				\hline
%				De 2 a 4 años & 100 - 120  \\
%				\hline
%				De 6 a 8 años & 100 - 115  \\
%				\hline
%				Adulto & 60 - 80  \\
%				\hline
%			\end{tabular}
%			\caption{Valores normales de frecuencia cardíaca.}
%		\end{center}
%	\end{table}
%	
%	\subsection{Frecuencia respiratoria}
%	Respiración es el término que se utiliza para indicar el intercambio de oxígeno y dióxido de carbono que se lleva a cabo en los pulmones y tejidos. \cite{cobo2011} El ciclo respiratorio comprende una fase de inspiración y otra de espiración:
%	\begin{itemize}
%		\item \textbf{Inspiración:} fase activa; se inicia con la contracción del diafragma y los músculos intercostales.
%		\item \textbf{Espiración:} fase pasiva; depende de la elasticidad pulmonar.
%	\end{itemize}
%	
%	La frecuencia respiratoria es el número de respiraciones que suceden en un minuto, y comprende el proceso de inhalación y exhalación. \cite{aguayoChile}\\
%	
%	La frecuencia se mide por lo general cuando una persona está en reposo y consiste simplemente en contar la cantidad de respiraciones durante un minuto cada vez que se eleva el pecho. La frecuencia respiratoria puede aumentar con la fiebre, las enfermedades y otras afecciones médicas. \cite{valoresUNAM}
%	
%	\begin{table}[htbp]
%		\begin{center}
%			\begin{tabular}{|l|l|}
%				\hline
%				\textbf{Edad} & \textbf{Respiraciones por minuto} \\
%				\hline \hline
%				Recién nacido & 30 - 80  \\
%				\hline
%				Lactante menor & 20 - 40  \\
%				\hline
%				Lactante mayor & 20- 30  \\
%				\hline
%				De 2 a 4 años & 20- 30  \\
%				\hline
%				De 6 a 8 años & 20 - 25  \\
%				\hline
%				Adulto & 12 -20  \\
%				\hline
%			\end{tabular}
%			\caption{Valores normales de frecuencia respiratoria.}
%		\end{center}
%	\end{table}
%	
%	\subsection{Presión arterial}
%	Es una medida de la presión que ejerce la sangre sobre las paredes arteriales en su impulso a través de las arterias. Debido a que la sangre se mueve en forma de ondas, existen dos tipos de medidas de presión: la presión sistólica, que es la presión de la sangre debida a la contracción de los ventrículos, es decir, la presión máxima; y la presión diastólica, que es la presión que queda cuando los ventrículos se relajan; ésta es la presión mínima. Tanto la presión sistólica como la diastólica se registran en "mm de Hg" (milímetros de mercurio). \cite{valoresUNAM} \cite{aguayoChile} \cite{signosvitales2016} \\
%	
%	La Presión arterial media se calcula con la siguiente fórmula: 
%	
%	\begin{center}
%		\textit{PA = Presión sistólica – Presión diastólica / 3 + Presión diastólica.}
%	\end{center}
%	
%	\begin{table}[htbp]
%		\begin{center}
%			\begin{tabular}{|l|l|l|}
%				\hline
%				\textbf{Edad} & \textbf{Presión Sistólica (mmHg)} & \textbf{Presión Diastólica (mmHg)} \\
%				\hline \hline
%				Lactante menor & 60 - 90 & 30 - 60 \\
%				\hline
%				2 años  & 78 - 112 & 48 - 78 \\
%				\hline
%				4 años & 85 - 114 & 52 - 85 \\
%				\hline
%				8 años & 95 - 135 & 58 - 88 \\
%				\hline
%				Adulto & 100 - 140 & 60 - 90 \\
%				\hline
%			\end{tabular}
%			\caption{Valores normales de la presión arterial.}
%		\end{center}
%	\end{table}
%	
%	\subsection{Temperatura}
%	La temperatura corporal representa el estado térmico del organismo y expresa el balance entre la producción de calor en el cuerpo (termogénesis) y la pérdida (termólisis). En el hombre, un conjunto de funciones fisiológicas contribuye a mantener constante la temperatura, que se mide por medio de termómetros. La temperatura normal del cuerpo varía según el sexo, la actividad reciente, el consumo de alimentos y líquidos, la hora del día y, en las mujeres, la etapa del ciclo menstrual. \cite{cobo2011} \cite{signosvitales2016}
%	
%	\begin{table}[htbp]
%		\begin{center}
%			\begin{tabular}{|l|l|}
%				\hline
%				\textbf{Edad} & \textbf{Grados Celsius} \\
%				\hline \hline
%				Recién nacido & 36.1 - 37.7  \\
%				\hline
%				Lactante & 37.2  \\
%				\hline
%				De 2 a 8 años & 37.0  \\
%				\hline
%				Adulto & 36.0 - 37.0  \\
%				\hline
%			\end{tabular}
%			\caption{Valores normales de temperatura.}
%		\end{center}
%	\end{table}