\section{Sensor de temperatura}

El sensor de temperatura MAX30205 tiene una interfaz de comunicación I2C, para poder comprobar su funcionamiento de forma unitaria, fue necesario configurar la interfaz I2C con la que cuenta el dsPIC30205 y adicionalmente crear las rutinas para mantener la comunicación entre ambos componentes. En la figura x se muestra en diagrama de flujo con el procedimiento realizado para esta prueba.\\

{\Large \TODO{Agregar imagen del diagrama de flujo.}}

El microcontrolador fue configurado como maestro y el sensor como esclavo, por lo que el maestro debe iniciar la comunicación, recibir los datos del sensor y terminar la comunicación. El código de las rutinas creadas para las operaciones anteriores se muestra a continuación.\\

Rutina que genera la condición de \textbf{start} al bus I2C.
{\small
\begin{lstlisting}[frame=single]
_START_I2C:
	BCLR	IFS0,	#MI2CIF
	BSET	I2CCON,	#SEN
ESPERA_START:
	BTSS	IFS0,	#MI2CIF
	GOTO	ESPERA_START
RETURN
\end{lstlisting}
}


Rutina que manda un dato de 8 bits al sensor de temperatura que es el dispositivo esclavo.	
{\small
\begin{lstlisting}[frame=single]

ENVIA_DATO_I2C:
	BCLR	IFS0,	#MI2CIF
	
	MOV.B	WREG,	I2CTRN
ESPERA_ENVIA_DATO_I2C:
	BTSS	IFS0,	#MI2CIF
	GOTO	ESPERA_ENVIA_DATO_I2C
	RETURN
\end{lstlisting}
}

Rutina que recibe un dato de 8 bits del dispositivo esclavo que es el sensor de temperatura.
{\small
\begin{lstlisting}[frame=single]
_RECIBE_DATO_I2C:
	BCLR	IFS0,	#MI2CIF
	BSET	I2CCON,	#RCEN
ESPERA_RECIBE_DATO_I2C:
	BTSS	IFS0,	#MI2CIF
	GOTO	ESPERA_RECIBE_DATO_I2C
	MOV		I2CRCV,	W0	
	RETURN
\end{lstlisting}
}

Después de realizar el programa, se conectó el sensor de temperatura al microcontrolador a través de las terminales SCL y SDA del mikrobus2 como se muestra en las figuras \ref{fig:ConexionMAX30205} y \ref{fig:ConexionFisicaMAX30205}.\\

\begin{figure}[htbp!]
	\centering
	\fbox{\includegraphics[width=0.8\textwidth]{AvancesPruebas/imagenes/MAX30205ConexionFisica.jpg}}
	\caption{Conexión física del sensor MAX30205.}
	\label{fig:ConexionFisicaMAX30205}
\end{figure}

\begin{figure}[htbp!]
	\centering
	\fbox{\includegraphics[width=0.7\textwidth]{AvancesPruebas/imagenes/MAX30205Conexion.png}}
	\caption{Conexión del sensor MAX30205.}
	\label{fig:ConexionMAX30205}
\end{figure}

Para visualizar los datos recibidos del sensor, se habilitaron las terminales del UART1 y se conectó el módulo FT232 a la computadora. El resultado de la ejecución del programa con las mediciones leídas se muestra en la figura \ref{fig:ResultadoTerminalMAX30205}.\\


\begin{figure}[htbp!]
	\centering
	\fbox{\includegraphics[width=0.7\textwidth]{AvancesPruebas/imagenes/MAX30205Conexion.png}}
	\caption{Conexión del sensor MAX30205.}
	\label{fig:ResultadoTerminalMAX30205}
\end{figure}

El resultado de esta prueba muestra que el sensor de temperatura MAX30205 funciona de manera esperada y la elección realizada fue correcta.

\clearpage