%========ANÁLISIS DE FACTIBILIDAD
\section{Analísis de Factibilidad}
\subsection{Introducción}
Para el desarrollo de aplicaciones moviles se encuentran diferentes herramientas para llevarlas acabo dentro de las cuales destacan :
Librerias ,Lenguajes de programacion o Arquitecturas.
Por lo cual se debe de definir un modelo el cual tenga un nivel alto de interoperabilidad entre los componentes del sitema.


\subsection{Factibilidad Tecnica}
La utilización de un analísis de factibilidad tecnica nos permite recolectar información necesaria sobre los recursos de software , hardware , conocimientos y habilidades de los cuales se puede hacer uso dentro del desarrollo e implementación del trabajo, no obstante el analísis de factibilidad tambien nos permite definir con mayor claridad los requerimientos tecnológicos que deben de ser utilizados para que el desarrollo tenga el menor indice de riesgo posible.


\paragraph{1.-Tipos de aplicaciones:}

Existen 3 tipos diferentes de aplicaciones movil , los factores en los cuales se basan para clasificarlas dependen de la funcionalidad y la forma en que seran realizadas. Los puntos mas destacables a la hora de una eleccion de aplicacion movil es sin duda la red de internet , ya que las aplicaciones moviles pueden trabajar con red de internet o se pueden ejecutar sin problemas desde el dispositivo sin contar con una red de internet.
A continuacion se explicara en una breve tabla las ventajas y desventajas que estas tienen:
\begin{table}[h!]
\begin{tabular}{|p{2.5cm}|p{5cm}|p{4cm}|p{4cm}|}
\hline
\textbf{Tipos de aplicaciones moviles}&\textbf{Descripción}& \textbf{Ventajas}& \textbf{Desventajas}\\
\hline
\hline
AppsNativas& Una aplicación nativa, se caracteriza por haber sido desarrollada especialmente para un lenguaje de programación.&
\begin{UClist}
		\UCli Notificaciones push
		\UCli Rapidez de navegación
		\UCli Uso complejo del hardware
		\UCli Facil Manejo
		\UCli Diseño y desarrollo Personalizado
		\UCli No necesitan conexión a internet
\end{UClist} & \begin{UClist}

\UCli Mayor tiempo de desarrollo
\UCli Precio mas elevado
\end{UClist} \\
\hline
\hline
WebApss&Este tipo de aplicaciones se caracterizan por estar desarrolladas en lenguajes de programación propias de la web, como HTML, CSS o Javascript.
& 	\begin{UClist}
		\UCli Multiplataformas
		\UCli Responsive Desing
	\end{UClist} &
	\begin{UClist}
		\UCli No requieren instalación por lo tanto no estaran disponibles en las tiendas de aplicaciones (Play Store , Apple Store).
		\UCli El diseño es totalmente web , "Responsive Desing" lo adapta a una plataforma movil.
	\end{UClist} \\
\hline
\hline
WebApps Nativas & Las Webapps Nativas es una combinacion de Webapps y appsnativas las cuales tienen la mayor parte de su estructura realiza en lenguajes de programación de la web , no dejando aun lado la parte nativa del movil. & 
	\begin{UClist}
		\UCli Multiplataforma
		\UCli Herramientas de lenguajes nativos
	\end{UClist} &
	\begin{UClist}
		\UCli Costo Elevado
	\end{UClist} \\
\hline



\end{tabular}
\caption{Tipos de aplicaciones.}
\label{disenoEstructura:TipoApp}
\end{table}

Para este trabajo terminal se hara uso total de las aplicaciones nativas ya que estas se centran en un solo lenguaje de programación ayudando asi a reducir considerablemente el tiempo de producción de la aplicación movil.



\paragraph{2. Software :}
En el mundo del desarrollo para móviles actualmente se dispone de muchas opciones para ello se debe tomar en cuenta los lenguajes y herramientas nativos de cada plataforma.Para ello se hizo un analisis de las herramientas que estan disponibles para la realización de aplicaciones moviles.

\begin{table}[h!]
\begin{tabular}{|p{3cm}|p{3.5cm}|p{3cm}|p{3cm}|p{3cm}|}
\hline
\textbf{Herramienta}&\textbf{Ventajas}& \textbf{Desventajas}& \textbf{Lenguaje Utilizado} & \textbf{Tipo de Herramienta} \\
Xamarin & 	\begin{UClist}
				\UCli Basada en lenguaje mas populares
				\UCli Es universal para las diferentes plataformas
			\end{UClist} &
							\begin{UClist}
								\UCli Mayor tiempo de desarrollo
							\end{UClist} & C\# , .NET & Multiplataforma compilado a nativo \\
\hline
\hline
AndroiStudio & 	\begin{UClist}
					\UCli Compilación rapida
					\UCli Ejecución de aplicación en tiempo real
					\UCli Ejecución de aplicación directamente desde el movil
					\UCli Renderizado en tiempo real
					\UCli Uso de parametros tools
				\end{UClist} &
								\begin{UClist}
									\UCli Los requisitos minimos de hardware son elevados.
									\UCli Se requiere una computadora de gama alta
								\end{UClist} & Android & Nativo \\
\hline
\hline
Swift & \begin{UClist}
			\UCli Proceso de desarrollo rapido.
			\UCli Escalable al producto y al equipo.
			\UCli Seguridad y rendimiento.
			\UCli Disminución de la huella de memoria
		\end{UClist} & 
						\begin{UClist}
							\UCli Lenguaje joven.
							\UCli Interoperabilidad con herramientas de terceros e IDE.
							\UCli Falta de soporte para versiones anteriores
						\end{UClist} & IOS & Nativo \\
\hline


\end{tabular}
\caption{IDE}
\label{disenoEstructura:IDE}
\end{table}

Despues de el analísis correspondiente se ha concluido que android studio es la mejor herramienta para poder realizar la aplicación movil
ya que cuenta con uno de los mejores emuladores para realizar las pruebas correspondientes a la etapa de testing.

\paragraph{3. Sistema Operativo} 
Debido al desarrollo a una aplicación nativa , se hara uso de un sistema operativo en especial para que asi al momento de generar la implementación correspondiente no se obtenga fallos de compatibilidad con el sistema operativo . En la tabla \ref{disenoEstructura:SO} se muestra los sistemas operativos como tambien las versiones que serán evaluados para determinar cual se debe de utilizar.

\begin{table}[htbp]
\begin{tabular}{|p{3cm}|p{12cm}|}
\hline
\textbf{Nombre}&\textbf{Descripción}\\
\hline
\hline
GNU/Linux & El sistema operativo Linux es considera un sistema operativo libre tipo Unix , del cual destacan sus caracteristicas como la multiplataforma, multiusuario y multitareas. Ademas el sistema operativo Linux es considerado como uno de los sitemas operativos mayormente soportados por la alta iteroperabilidad que tiene con las principales plataformas informaticas.  \\
\hline
\hline
Windows 8.1® & Windows 8.1 ® perteneciente a la gran cadena de microsoft es una actualización al antiguo windows 8® . El cual te permite sincronizar mas configuraciones entre dispositivos , dentro de las configuraciones que mas destacan es la posibilidad de poder configurar la pantalla de inicio , configuración de teclado y raton por medio de bluetooth . Todo esto como parte de la expansión de configuracion PC.\\
\hline
\hline
Windows 10® & Windows 10® Es considerado el ultimo sistema operativo desarrollado por Microsoft® hasta la fecha actual , este sistema operativo pertenece a la familia de sistemas operativos NT. Dentro de los aspectos mas destacables que poseé este sistema operativo es la capacidad de tener una mejor armonización entre usuario y funcionalidad , ademas de corregir las deficiencias en las interfaces anteriores que se vienen presentando desde la presentacion de Windows 8® en el mercado   \\
\hline
\hline
Mojave® & La franquicia Apple ® tiene dentro de su mercado de ordenadores la ultima actualización conocida como Mojave que quita extensiones de aplicaciones que esten desarrolladas a 32 bits , esta actualizacioón presenta mejoras en la interfaz grafica para una mejor experiencia del usuario .  \\
\hline
\end{tabular}
\caption{SO}
\label{disenoEstructura:SO}
\end{table}

Una vez terminado el analísis se concluyo que se hara uso del sistema operativo mas actualizado enfocado a un mejor rendimiento de la aplicación web durante el desarrollo e implementación del mismo por lo cual se utilizara el sistema operativo Windows 10®.





\paragraph{4. Lenguaje de Programación} 
Para el desarrollo de aplicaciones movil, se manejan lenguajes  que son capaces de  ser interpretados por la mayoria de dispositivos moviles android el cual da la lógica a la aplicación, existen una gran variedad de librerías las cuales facilitan el desarrollo movil.
%En la tabla \ref{disenoEstructura:lenguajes} se muestran describen los lenguajes de programación que serán usados dependiendo de \\
%
%
%\begin{table}[htbp]
%	\begin{center}
%		\begin{tabular}{|p{3cm}|p{12cm}|}
%			\hline
%			%			\rowcolor{colorSecundario}
%			%			\color{green}
%			\thead{Nombre}&\thead{Descripción}\\
%			\hline
%			\hline
%			C &  C es un lenguaje de programación de tipos de datos estáticos, débilmente tipificado, de medio nivel, ya que dispone de las estructuras típicas de los lenguajes de alto nivel pero, a su vez, dispone de construcciones del lenguaje que permiten un control a muy bajo nivel. Los compiladores suelen ofrecer extensiones al lenguaje que posibilitan mezclar código en ensamblador con código C o acceder directamente a memoria o dispositivos periféricos.\\
%			\hline
%			Ensamblador & El lenguaje ensamblador es un lenguaje de programación de bajo nivel. Consiste en un conjunto de mnemónicos que representan instrucciones básicas para los computadores, microprocesadores, microcontroladores y otros circuitos integrados programables. Implementa una representación simbólica de los códigos de máquina binarios y otras constantes necesarias para programar una arquitectura de procesador y constituye la representación más directa del código máquina específico para cada arquitectura legible por un programador.\\
%			\hline
%			Java & Java es un lenguaje de programación de propósito general, concurrente, orientado a objetos, que fue diseñado específicamente para tener tan pocas dependencias de implementación como fuera posible. Su intención es permitir que los desarrolladores de aplicaciones escriban el programa una vez y lo ejecuten en cualquier dispositivo, lo que quiere decir que el código que es ejecutado en una plataforma no tiene que ser recompilado para correr en otra.
%			Java fue elegido como el lenguaje para el entorno de desarrollo de Android.\\
%			\hline
%			M & Las aplicaciones de MATLAB se desarrollan en un lenguaje de programación propio. Este lenguaje es interpretado, y puede ejecutarse tanto en el entorno interactivo, como a través de un archivo de script (archivos *.m). Este lenguaje permite operaciones de vectores y matrices, funciones, cálculo lambda, y programación orientada a objetos.\\
%			\hline
%		\end{tabular}
%		\caption{Lenguajes de programación}
%		\label{disenoEstructura:lenguajes}
%	\end{center}
%\end{table}
%
%Debido a la naturaleza del presente trabajo terminal, es necesario el uso de más de un lenguaje de programación. Para la programación del microcontrolador, se hará uso de los lenguajes C y Ensamblador, los cuales son soportados por el entorno de desarrollo integrado MPLAB X. Adicionalmente, se hará uso del lenguaje java, implementado para el desarrollo de la aplicación móvil a través de Android Studio. Y para la ejecución de las pruebas unitarias, el lenguaje utilizado será M, en el IDE de MATLAB.
%
%
%%\paragraph{4. Gestor de Base de Datos} \textcolor{White}{.} \newline

\subsubsection{Hardware}
%Evaluando el equipo de cómputo con el que se cuenta y analizando las especificaciones y requisitos de los entornos de desarrollo elegidos, no es necesario realizar una inversión inicial para la adquisición de nuevos equipos pues las características con las que cuentan, satisfacen los requerimientos necesarios para el desarrollo del trabajo. Sin embargo, para el hardware que será implementado en el sistema embebido, sí es necesario realizar una inversión inicial por el costo unitario de cada uno.\\
%
%En la tabla \ref{disenoEstructura:equipos} se describen las especificaciones de los equipos que se tienen y que serán utilizados para el diseño, desarrollo e implementación del trabajo.
%
%\begin{table}[htbp]
%	\begin{center}
%		\begin{tabular}{|c|p{5cm}|p{7cm}|}
%			\hline
%			%			\rowcolor{colorSecundario}
%			%			\color{green}
%			Cantidad&Equipo&Características\\
%			\hline
%			\hline
%			1 & Sony VAIO Pro SVP132A1CU & \begin{UClist}
%				\UCli Procesador: Intel® Core i5-4200U CPU $@$ 1.60 GHz x 4
%				\UCli RAM: 8GB
%				\UCli Sistema Operativo: Windows 10 64-bit / Zorin OS 64bit
%				\end{UClist} \\
%			\hline
%			1 & Asus VivoBook S510U & \begin{UClist}
%				\UCli Procesador: Intel® Core i7-8550U CPU $@$ 1.80 GHz x 8
%				\UCli RAM: 8GB
%				\UCli Sistema Operativo: Windows 10 64-bit / Zorin OS 64bit
%			\end{UClist}\\
%			\hline 
%			1 & Smartphone Motorola G3 & \begin{UClist}
%				\UCli Procesador: Snapdragon 410
%				\UCli RAM: 1GB
%				\UCli Sistema Operativo: Android 6.0 Mashmallow
%			\end{UClist}\\
%			\hline
%			1 & Smartphone Xiaomi Mi A2 &\begin{UClist}
%				\UCli Procesador: Snapdragon 660
%				\UCli RAM: 4GB
%				\UCli Sistema Operativo: Android 8.1 Oreo con Android One
%			\end{UClist}\\
%			\hline
%		\end{tabular}
%		\caption{Características del equipo disponible}
%		\label{disenoEstructura:equipos}
%	\end{center}
%\end{table}
%
%En la tabla \ref{disenoEstructura:recursosHardware} se listan los recursos de hardware que, según el análisis realizado en el capítulo \ref{chp:analisis} es necesario adquirir para el diseño, desarrollo e implementación del trabajo.\\
%
%\begin{table}[htbp]
%	\begin{center}
%		\begin{tabular}{|c|p{10cm}|}
%			\hline
%			%			\rowcolor{colorSecundario}
%			%			\color{green}
%			\thead{Cantidad}&\thead{Recurso}\\
%			\hline
%			\hline
%			1 & Microcontrolador dsPIC30F4013 \\
%			\hline
%			1 & Módulo GSM \\
%			\hline
%			1 & SIM telefónico para módulo GSM\\
%			\hline
%			1 & Sensor de temperatura MAX30205\\
%			\hline
%			1 & Sensor de pulso\\
%			\hline
%			1 & Programador para microcontrolador\\
%			\hline
%			1 & Módulo FT232 \\
%			\hline
%		\end{tabular}
%		\caption{Recursos de hardware necesarios}
%		\label{disenoEstructura:recursosHardware}
%	\end{center}
%\end{table}
%
%Como resultado del análisis de factibilidad técnica, se determinó que se cuenta con la infraestructura tecnológica necesaria para el desarrollo del trabajo terminal.
%
%%==============================================ECONÓMICA==================================================
%\newpage
\subsection{Factibilidad Económica}
%El análisis de la factibilidad económica determina si los recursos económicos y financieros  son suficientes para llevar a cabo las actividades o procesos, además permite conocer los costos estimados para el desarrollo del trabajo.\\
%
%Para determinar la factibilidad económica del presente trabajo terminal se realizó un análisis describiendo los gastos totales, los cuales fueron clasificados en las siguientes categorías:
%
%\begin{enumerate}
%	\item Gastos tecnológicos
%	\item Gastos por servicios
%\end{enumerate}
%
%Debido a este proyecto tiene fines académicos, en los gastos descritos no se consideran sueldos.
%

\paragraph{1. Gastos tecnológicos} \textcolor{White}{.} \newline
%En esta sección se consideran todos los gastos relacionados con el software y hardware implementado en el desarrollo del proyecto. Todas las cantidades descritas están consideradas en pesos mexicanos.\\
%
%En la tabla \ref{disenoEstructura:gastosDepreciacion} se describen los gastos del equipo que fue utilizado pero que no se realizó una inversión inicial pues ya se contaba con él. Para estos equipos se consideró únicamente el gasto de la depreciación durante los meses de trabajo. El porcentaje de depreciación anual que se consideró para todos los equipos fue del 20\% y el tiempo de trabajo en meses estimado para este trabajo terminal fue de 9.
%
%\begin{table}[htbp]
%	\begin{center}
%		\begin{tabular}{|p{4cm}|p{3cm}|p{3cm}|p{3cm}|}
%			\hline
%			%			\rowcolor{colorSecundario}
%			%			\color{green}
%			\thead{Equipo}&\thead{Precio de compra\\(\$)}&\thead{Depreciación mensual\\(\$)}&\thead{Depreciación total\\(\$)} \\
%			\hline
%			\hline
%			Sony VAIO Pro SVP132A1CU &15,000.00 &250.00&2,250.00 \\
%			\hline
%			Asus VivoBook S510U & 24,000.00&400.00&3,600.00 \\
%			\hline
%			Smartphone Motorola G3 &3,000.00 &50.00&450.00\\
%			\hline
%			Smartphone Xiaomi Mi A2 & 5,000.00&83.33&750.00\\
%			\hline
%			\hline
%			TOTAL & &&7,050.00\\
%			\hline
%		\end{tabular}
%		\caption{Gastos tecnológicos por depreciación}
%		\label{disenoEstructura:gastosDepreciacion}
%	\end{center}
%\end{table}
%
%En la tabla \ref{disenoEstructura:gastosHardware} se describen los gastos relacionados con la compra del equipo de hardware específico para el desarrollo del sistema embebido.
%
%\begin{table}[htbp]
%	\begin{center}
%		\begin{tabular}{|c|p{4cm}|p{3cm}|p{3cm}|}
%			\hline
%			%			\rowcolor{colorSecundario}
%			%			\color{green}
%			\thead{Cantidad}&\thead{Recurso}&\thead{Precio unitario\\(\$)}&\thead{Subtotal\\(\$)} \\
%			\hline
%			\hline
%			1 &Microcontrolador dsPIC30F4013&143.39&143.39\\
%			\hline
%			1 & Módulo GSM&1,072.00&1,072.00 \\
%			\hline
%			1 &SIM telefónico para módulo GSM&0.00&0.00\\
%			\hline
%			1 &Sensor de temperatura MAX30205&30.57&30.57\\
%			\hline
%			1 &Sensor de pulso&477.46&477.46\\
%			\hline
%			1 &Programador para microcontrolador&300.00&300.00\\
%			\hline
%			1 &Módulo FT232&100.00&100.00\\
%			\hline
%			\hline
%			TOTAL & &&2,123.42\\
%			\hline
%		\end{tabular}
%		\caption{Gastos tecnológicos por hardware}
%		\label{disenoEstructura:gastosHardware}
%	\end{center}
%\end{table}


\paragraph{2. Gastos por servicios} \textcolor{White}{.} \newline
%El desarrollo del trabajo implica gastos para los servicios con los que funcionarán los recursos de hardware y software mencionados anteriormente. El gasto estimado por los servicios a utilizar se describe en la tabla \ref{disenoEstructura:gastosServicios}.\\
%
%
%\begin{table}[htbp]
%	\begin{center}
%		\begin{tabular}{|p{4cm}|p{3cm}|p{3cm}|}
%			\hline
%			%			\rowcolor{colorSecundario}
%			%			\color{green}
%			\thead{Servicio}&\thead{Gasto mensual\\(\$)}&\thead{Gasto total\\(\$)}\\
%			\hline
%			\hline
%			Internet &150.00 &1,350.00 \\
%			\hline
%			Luz eléctrica &135.00 &1,215.00 \\
%			\hline
%			SIM telefónica &50.00 &450.00\\
%			\hline
%			\hline
%			TOTAL & &3,015\\
%			\hline
%		\end{tabular}
%		\caption{Gastos por servicios}
%		\label{disenoEstructura:gastosServicios}
%	\end{center}
%\end{table}

\paragraph{Gastos totales} \textcolor{White}{.} \newline
%Para obtener el monto total de los gastos para el proyecto, se sumaron los gastos mencionados anteriormente, como se muestra en la tabla \ref{disenoEstructura:gastosTotales}, por lo tanto el costo total estimado del proyecto es: \$9,176  pesos mexicanos.
%
%\begin{table}[htbp]
%	\begin{center}
%		\begin{tabular}{|p{4cm}|p{4cm}|}
%			\hline
%			%			\rowcolor{colorSecundario}
%			%			\color{green}
%			\thead{Concepto}&\thead{Gasto\\(\$)}\\
%			\hline
%			\hline
%			Gastos tecnológicos &  7,050.00\\
%			\hline
%			Gastos por servicios & 3,015.00\\
%			\hline
%			\hline
%			TOTAL & 9,176.43\\
%			\hline
%		\end{tabular}
%		\caption{Gastos totales}
%		\label{disenoEstructura:gastosTotales}
%	\end{center}
%\end{table}
%  

