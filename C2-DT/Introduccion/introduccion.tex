%%%%%   INTRODUCCIÓN   %%%%%
%---------------------------------------------------------
%\section{Objetivo del documento}
%    El presente documento \varCveDocumento\ tiene como objetivo mostrar en detalle los requerimientos funcionales y no funcionales, modelos de información, reglas de negocio, modelos de comportamiento, casos de uso e interfaces correspondientes a cada uno de los módulos propuestos para esta etapa.

\section{Planteamiento del problema}
De acuerdo con datos del Instituto Nacional de Estadística y Geografía (INEGI), tan solo en México en el 2017 se registraron 241,285 colisiones con vehículos automotores y 34,910 colisiones con motocicletas, de las cuales 258,168 fueron causadas por el conductor.\\ 
%
De acuerdo con cifras de 2017 de la Organización Mundial de la Salud (OMS), cada año mueren en el mundo cerca de 1,3 millones de personas en accidentes de tránsito, y entre 20 y 50 millones padecen traumatismos no mortales causantes de discapacidad. Los accidentes viales, además, constituyen una de las principales causas de mortalidad en todos los grupos etarios, principalmente entre personas de entre 15 y 19 años.\\

De acuerdo con datos del Instituto Nacional de Salud Pública (INSP), nuestro país ocupa el séptimo lugar a nivel mundial y el tercero en la región de Latinoamérica en muertes por siniestros viales, con 22 decesos de jóvenes de entre 15 y 29 años al día, y 24 mil decesos en promedio al año. Los siniestros viales constituyen la primera causa de muerte en jóvenes entre 5 y 29 años de edad y la quinta entre la población general.\\

En México opera un número único de emergencias 911; en él se homologan todos los números de atención de emergencias médicas, de seguridad y de protección civil a nivel federal, estatal y municipal. En consecuencia, existe una necesidad de utilizar la tecnología para la comunicación directa con servicios de emergencia específicos, así como con contactos de confianza y/o aseguradoras de la persona afectada, esto sin la necesidad de que el afectado lo solicite. Asimismo cuando una persona no esta en sus facultades para poder dar información sobre el suceso, entorpece la logística para poder ser auxiliado. 



\section{Solución propuesta}

Por lo anterior, hemos decidido crear un sistema que facilite el monitoreo del posible estado de un automóvil haciendo uso de los sensores con los que cuenta y los de un dispositivo móvil (smartphone) para que con los datos medidos e interpretados se pueda informar en caso de un posible percance automovilístico a las personas interesadas, como familiares, aseguradoras o cuerpos de emergencia, y en el caso de que el usuario afectado necesite auxilio, tratar la emergencia lo más pronto posible.



\section{Justificación}
Hoy la tecnología se ha visto integrada en casi todas las actividades cotidianas, con la implemetación de la Internet de las cosas, donde el uso de las tecnologías de información y comunicación (TIC) ha permitido la integración de los dispositivos para la medición y manipulación de los dispositivos de forma remota.\\



Lo que se busca con este Trabajo Terminal es crear un sistema que permita almacenar e interpretar la información del posible estado de un automóvil para poder notificar de forma automática y manual un posible percance automovilístico, esto en el caso de que los tripulantes no tengan la capacidad físicas para hacerlo.\\


La propuesta comprenderá un dispositivo electrónico comunicado con uno o varios sensores para el monitoreo del posible estado de un automóvil que con ayuda de los sensores integrados en el dispositivo móvil (smartphone) permitir la recopilación y procesamiento de la información obtenida, la cual será almacenada y en caso de que la interpretación de dichos datos aparenté ser un un percance automovilístico, notificar a los contactos como cuerpos de emergencia, familiares, aseguradoras etc.



\section{Alcance}

\section{Objetivos}
\subsection{Objetivo general}
Analizar y desarrollar una aplicación móvil que alerte de manera automática y/o manual, durante un percance automovilístico a servicios de emergencia , familiares directos y/o contactos registrados por el usuario.
%Implementar un sistema embebido que permita el monitoreo remoto de signos vitales de frecuencia cardíaca y temperatura corporal usando IoT.
\subsection{Objetivos específicos}
\begin{itemize}
	\item Disminuir el tiempo de notificación a hospitales y/o autoridades por percances automovilísticos
	\item Temporizador de alerta SOS para percances automovilísticos de bajo grado
	\item Gestionar contactos de emergencia que recibirán el llamado SOS
	\item Ubicar el servicio de emergencia mas cercano a la colisión automovilística
	\item Confirmar la lectura correspondiente a la alerta SOS enviada a los familiares y/o servicios de emergencia.
	\item Notificar en tiempo real la ubicación donde ocurrió la colisión automovilística.
\end{itemize}
%	\begin{itemize}
%%		\item Realizar la configuración del sensor de pulso y la adquisición de la señal.
%%		\item Realizar la configuración del sensor de temperatura y la adquisición de la señal.
%%		\item Realizar la configuración del módulo de comunicación entre el sistema y el teléfono celular.
%%		\item Diseñar y construir una aplicación móvil para la consulta de los valores enviados al teléfono celular.
%%		\item Diseñar y construir un sistema embebido para la medición y procesamiento de la frecuencia cardíaca y la temperatura.
%	\end{itemize}	

%\section{Metodología}
%Para la realización del trabajo terminal se propone emplear el Modelo en V ya que ofrece una visión detallada de los diversos pasos e interacciones relacionados con el proceso de desarrollo y puede considerarse como un flujo de trabajo comúnmente utilizado. En la Figura \ref{fig:IntroduccionMetodologia} se muestran las principales actividades abordadas por el método. Convencionalmente, el lado izquierdo del modelo representa las fases del diseño del sistema, mientras que el lado derecho representa las fases de validación y verificación del sistema integrado.
%
%%\TODO \textbf{Cambiar resolución de imagen.}
%\begin{figure}[htbp!]
%	\centering
%	\fbox{\includegraphics[width=\textwidth]{Introduccion/imagenes/metodologia.png}}
%	\caption{Fases del modelo en V.}
%	\label{fig:IntroduccionMetodologia}
%\end{figure}
%
%El desarrollo se llevará a cabo en las siguientes etapas:
%
%\begin{enumerate}
%	\item Análisis de requerimientos. Esta fase consiste en establecer qué debe hacer el sistema ideal, sin determinar cómo se construirá o diseñará el software. 
%	\item Diseño de arquitectura del sistema. El diseño de la arquitectura del sistema consiste en varios pasos, como refinar las funciones del sistema y asignarlas a los diferentes componentes del sistema que pueden ser físicos o de hardware.
%	\item Diseño de arquitectura de SW y HW. En esta fase del desarrollo del sistema, se diseña el hardware y el software de los diversos elementos que constituyen los componentes del sistema global. Las actividades que se aplican son similares a las realizadas en la fase anterior, pero centrándose en un componente específico del sistema: 
%		\begin{itemize}
%			\item Refinamiento de los requerimientos funcionales y no funcionales del hardware y software.
%			\item Asignación de las funciones del software a los componentes de hardware.
%		\end{itemize}
%	\item Desarrollo del SW y Construcción del HW. Una vez que todos los componentes del sistema están diseñados, los elementos de hardware se construyen físicamente y los módulos de software son desarrollados en paralelo, y finalmente integrados con el hardware. Al final de este paso, los elementos de software y hardware deben estar disponibles para las actividades de verificación. Pueden realizarse algunas pruebas unitarias en paralelo con la implementación.
%	\item Integración y verificación del SW y HW. En este paso se ensamblan los componentes de hardware y software. Las pruebas de verificación se ejecutan para comprobar el cumplimiento de los objetivos de diseño.
%	\item Integración del sistema y prueba de verificación. En este paso, los elementos del sistema (HW, SW) se combinan y tiene lugar la verificación de los requisitos del sistema. 
%	\item Prueba del sistema y validación. Esta última de verificación tiene como objetivo validar si los resultados obtenidos cumplen con los requerimientos.
%\end{enumerate}

\cfinput{Introduccion/estadoArte}

%\section{Estructura del documento}
%	\textbf{Creo que estaría bien agregar la descripción de los capítulos que abarca el documento pero no sé si aquí esté bien.}