%=================ESTADO DEL ARTE=================
\section{Estado del arte}

\subsection{introducci�n}

A lo largo del tiempo los  accidentes automovilisticos han sido cada vez mas recurrentes y por ende han aumentado las muertes durante los sucesos. Por ello algunas empresas se han dedicado a tomar el problema como una prioridad que ayude a disminuir el indice de choques automoviliticos ,por medio de telefonos inteligentes , los cuales ayudan a tener una rapida interaccion entre el usuario y la aplicaci�n.
Nos hemos dado a la tarea a investigar las empresas que han contribuido a la realizacion de diferentes aplicaciones que pueden ayudar para este problema. Asi como tambien el costo que estas tienen en el mercado, los sensores o herramientas que utilizan para la creacion de dicha aplicaci�n y por ultimo el pais y sistema operativo al cual esta predestinadas.

\subsection{SOSmart}
SOSmart es una aplicación móvil que ofrece a sus usuarios notificaciones automáticas, así como una plataformaweb para el monitoreo, que permite monitorear en tiempo real emergencias activadas.\\

Notificación automática de accidentes.\\
SOmart detecta el accidente automáticamente utilizando los sensores internos del smartphone, y de manera inmediata envía notificación de emergencia con la ubicación a contactos de emergencia, seleccionados previamente.\\

Encendido manual o automático.\\
El servicio de detección de accidentes puede ser activado manualmente o configurado en su modo automático.
\begin{itemize}
	\item Modo automático: Cada vez que se detecte que estas en un vehículo en movimiento el algoritmo se encenderá automáticamente. Recomendado para quienes se desplazan en vehículo frecuentemente.
	
	\item Modo manual: Enciende el monitoreo de accidentes con un switch dentro de la app. Recomendado para quienes ocasionalmente viajan en vehículo.\\
\end{itemize}


Alarma efectiva a contactos.\\
Siempre y cuando los contactos cuenten con SOSmart instalado, al momento de notificar el accidente una alarma ruidosa se activará en su célular. De esta manera sin importar lo que estén haciendo serán notificados y podrán asistirte lo antes posible.\\

Botón de panico.\\
Para cualquier tipo de emergencia, basta con apretar nuestro botón de pánico y se informara de la emergencia, y ubicación a sus contactos previamente elegidos.\\


Algoritmo basado en datos reales.\\
El algoritmo de detección ha sido desarrollado usando datos de choques reales entregados por la National Highway Safety Asociation de Estados Unidos para detectar accidentes severos.  De esta manera nuestra aplicación es capaz de diferenciar cuando el Smartphone se ha caído, frenadas bruscas o choques menores, sin alertar de manera innecesaria.\\

Localización de hospitales cercanos.\\
En emergencia recibir atención médica lo más rapido posible puede hacer la diferencia. Es por ello que SOSmart, sin importar en el lugar del mundo que se encuentre el coche, te muestra una lista de hospitales cercanos y como llegar.\\

Sistema web de monitoreo para instituciones.\\
SOSmart ofrece una plataforma web de monitoreo, que permite a todo tipo de instituciones monitorear en tiempo real emergencias activadas por sus empleados y/o afiliados.  Este servicio es ampliamente usado por empresas de transporte, seguridad y hospitales.\\

\subsection{GAT}

GAT es una aplicación para móviles detecta accidentes y avisa a los servicios de emergencia.\\

GAT (Gestión de Accidentes de Tráfico) es el nombre de una aplicación desarrollada en el marco del Club Universitario de Innovación 2012, organizado por la Universidad Pontificia de Salamanca (UPSA), que pretende favorecer y acelerar la actuación de los servicios de emergencia en caso de accidente proporcionando la localización exacta del siniestro, así como los datos personales y médicos del propietario del teléfono.

La aplicación realiza un registro de los datos aportados por estos dos sensores dentro del uso normal del teléfono, y después provocaron accidentes a pequeña escala mediante un prototipo instalado en un coche teledirigido, obteniendo un patrón común a todos los accidentes. \\

De este modo, la aplicación puede discriminar entre una situación de accidente y una normal, aunque los autores han destacado que en esta fase de desarrollo es posible que la caída del teléfono pueda activar la aplicación, ya que la energía de un accidente con el coche radiocontrol es muy inferior a la que se produce en un choque con un vehículo real. \\

Detectado el accidente, la aplicación lanza una cuenta atrás de 20 segundos durante la cual se puede anular el envío del aviso. Si no se anula, se envían a los servicios de emergencia los datos personales y sanitarios del usuario y la localización GPS del dispositivo. Los datos que se introducen al configurar el programa son nombre, apellidos, DNI, grupo sanguíneo y alergias.\\

Problemas técnicos y legales:\\

Durante el desarrollo de la aplicación se han encontrado con varios problemas, tanto técnicos como legales. El principal problema técnico es que la plataforma Windows Phone (Plataforma en la que se encuentra desarrollada la aplicación) solicita siempre la confirmación del usuario para realizar cualquier envío de datos. Suponiendo que si el propietario del móvil ha sufrido un accidente éste puede estar incapacitado para activar la confirmación del envío, García Bellido y Honorato Morán han configurado un sistema para saltarse este paso. Un problema legal que aún deben solventar es que los datos personales y médicos no deben estar accesibles a terceros, por lo que deben enviarse encriptados.


%\TODO \textbf{Recuerdo que había que diferenciar entre los artículos y las tesis pero no sé cómo hacerlo. ¿Así está bien?} 
%\subsection{Tesis}
%	Diana Olvera y José Gonzáles proponen un Sistema de Monitoreo de Signos Vitales capaz de monitorear la presión arterial, el ritmo cardíaco y la temperatura corporal desde el mismo dispositivo con la finalidad de que estas mediciones puedan ser realizadas desde cualquier lugar  sin tener un amplio conocimiento en medicina para su uso. Este sistema se desarrolló implementando el PIC18F4550 para procesar las señales provenientes de los sensores y desplegarlos en un display LCD \cite{olvera2013}.\\
%	
%	En el “Sistema de monitoreo remoto y evaluación de signos vitales en pacientes con enfermedades crónicas”, López, Guerrero y Ramos proponen un sistema que sensa los signos vitales de un paciente y utiliza bluetooth y WiFi para transmitir dicha información a un dispositivo móvil encargado de evaluarla y alertar a alguna persona en caso de requerirse \cite{guerrero}.\\
%	
%	Existe un prototipo de hardware/software propuesto por Chávez, Martínez y Torres el cual mediante el uso de un sistema embebido dentro de un microcontrolador DSPIC30F3013 ayuda en el procesamiento, medición y envío mediante una red, tres signos vitales (presión arterial, frecuencia cardiaca y temperatura corporal), dichas mediciones son procesadas por el microcontrolador y enviadas mediante UART hacia un controlador Ethernet para ser recibidas y mostradas por una aplicación diseñada para el usuario \cite{ramirez2015}. \\
%
%\subsection{Artículos}
%	El prototipo diseñado por Li, Cummings, Lan, Graves y Wu en \cite{li2009}, es un sistema de monitoreo de la frecuencia cardiaca y respiratoria de bebés. Está compuesto por una unidad de monitoreo y una unidad receptora. Con un circuito de radiofrecuencia es capaz de producir y recibir señales de radio para la detección de los signos vitales sin la necesidad de que un sensor tenga contacto con el bebé. Utiliza un microcontrolador para procesar la señal recibida y un chip de comunicación XBee para la comunicación inalámbrica con el módulo receptor. \\
%	
%	En el artículo \cite{gitbau2011}, Girbau, Ramos , Lázaro y Villarino, propone un sistema para el monitoreo de los signos vitales utilizando un radar Doppler y la interfaz Zigbee para enviar la información del sensor de microondas. En este sistema se detecta la respiración y la frecuencia cardiaca, se adquiere con el sistema Zigbee y se transmite vía radiofrecuencia. Permite el monitoreo simultáneo de varias personas localizadas en diferentes nodos Zigbee. \\
%	
%	Cruz y Barros proponen en \cite{cruz2005} el uso de PDAs la adquisición de los signos vitales y su transmisión a un servidor de cuidados de la salud en donde un especialista pueda analizar el electrocardiograma (ECG) generado. Para la adquisición del ECG utiliza electrodos en el paciente. Y para la sincronización de los datos almacenados en la PDA con el servidor, requiere de una conexión a Internet.