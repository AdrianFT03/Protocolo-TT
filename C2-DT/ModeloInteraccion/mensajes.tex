%\DONE{} 
\section{Diseño de mensajes}
	En esta sección se describen los mensajes utilizados en el prototipo actual del sistema. Los mensajes se refieren a todos
	aquellos avisos que el sistema muestra al actor a través de la pantalla debido a diversas
	razones, por ejemplo: informar acerca de algún fallo en el sistema o para notificar acerca de alguna operación importante sobre la información.

%===========================================================removiendo los puntos generados
\subsection{Parámetros comunes}
    Cuando un mensaje es recurrente se parametrizan sus elementos, por ejemplo los mensajes: ``Aún no existen registros de {\em escuelas} en el sistema.'', ``Aún no existen registros de {\em responsables del programa} en el sistema.'', 
    ``Aún no existen registros de {\em integrantes de líneas de acción} en el sistema.'', tienen una estructura similar 
    por lo que para definir el mensaje se utilizan parámentros, con el objetivo de que el mensaje sea genérico y  
    pueda utilizarse en todos los casos que se considere necesario.\\
    
    Los parámetros también se utilizan cuando la redacción del mensaje tiene datos que son ingresados por el actor o que dependen del resultado de la operación, por ejemplo: 
    ``La {\em escuela  15DPR2497K} ha sido {\em modificada} exitosamente.''. En este caso la redacción se presenta parametrizada de la forma: ``DETERMINADO ENTIDAD VALOR ha sido OPERACIÓN exitosamente.'' y los 
    parámetros se describen de la siguiente forma:
    
    \begin{itemize}
	\item DETERMINADO ENTIDAD: Es un artículo determinado más el nombre de la entidad sobre la cual se realizó la acción.
	\item VALOR: Es el valor asignado al atributo de la entidad, generalmente es el nombre o la clave.
	\item OPERACIÓN: Es la acción que el actor solicitó realizar.
    \end{itemize}

    En el ejemplo anterior se hace referencia a VALOR, es decir: {\em 15DPR2497K} es el {\bf valor}  de la entidad {\bf escuela}. Cada mensaje enlista los parámetros 
    que utiliza, sin embargo aquí se definen los más comunes a fin de simplificar la descripción de los mensajes:

    \begin{description}
	\item [ARTÍCULO:] Se refiere a un {\em artículo} el cual puede ser DETERMINADO (El $\mid$ La $\mid$ Lo $\mid$ Los $\mid$ Las) o INDETERMINADO (Un $\mid$ Una $\mid$ 
	Uno $\mid$ Unos $\mid$Unas) se aplica generalmente sobre una ENTIDAD, ATRIBUTO o VALOR.
	\item [CAMPO:] Se refiere a un campo del formulario. Por lo regular es el nombre de un atributo en una entidad.
	\item [CONDICIÓN:] Define una expresión booleana cuyo resultado deriva en {\em falso} o {\em verdadero} y suele ser la causa del mensaje.
	\item [DATO:] Es un sustantivo y generalmente se refiere a un atributo de una entidad descrito en el modelo estructural del negocio, por ejemplo: número de incendio,
	brigada de apoyo del incendio, uso de suelo autorizado del predio, etc. %ATRIBUTO
	\item [ENTIDAD:] Es un sustantivo y generalmente se refiere a una entidad del modelo estructural del negocio, por ejemplo: incendio, pago por servicios ambientales hidrológicos, reforestación, etc.
	\item [OPERACIÓN:] Se refiere a una acción que se debe realizar sobre los datos de una o varias entidades. Por ejemplo: registrar, eliminar, actualizar, etc. Comúnmente 
	la OPERACIÓN va concatenada con el sustantivo, por ejemplo: Registro de un nuevo beneficio, registro de una actividad, eliminar una tarea, etc.
	\item [VALOR:] Es un sustantivo concreto y generalmente se refiere a un valor en específico. Por ejemplo: ``2014-003'', que es un valor concreto del DATO de la 
	ENTIDAD ``incendio''.
	\item [TAMAÑO:] Es el tamaño del atributo de una entidad, el cual se encuentra definido en el diccionario de datos.
	\item [MOTIVO:] Es una explicación acerca de la operación que se pretende realizar.
    \end{description}


\subsection{Mensajes a través de la pantalla}

%===========================  MSG1 ==================================
\begin{mensaje}{MSG1}{Operación realizada exitosamente}{Notificación}
    \item[Ubicación:] \msjSuperior
  %  \item[Estatus:] Terminado
    \item[Objetivo:] Notificar al actor que la acción solicitada fue realizada exitosamente.
    \item[Redacción:] DETERMINADO ENTIDAD VALOR ha sido OPERACIÓN exitosamente.
    \item[Parámetros:] El mensaje se muestra con base en los siguientes parámetros:
    \begin{Citemize}
	\item DETERMINADO ENTIDAD: Es un artículo determinado más el nombre de la entidad sobre la cual se realizó la acción.
	\item VALOR: Es el valor asignado al atributo de la entidad, generalmente es el nombre o la clave.
	\item OPERACIÓN: Es la acción que el actor solicitó realizar.
    \end{Citemize}
    \item[Ejemplo:] {\em El paciente Carlos Granados} ha sido {\em registrado} exitosamente.
    \item[Referenciado por:] \cdtIdRef{CU2}{Registrar paciente}, \cdtIdRef{CU4}{Editar información del paciente}
\end{mensaje}

%%===========================  MSG2 ==================================
%\begin{mensaje}{MSG2}{Falta dato obligatorio}{Error}
%	\item[Ubicación:] \msjCampo
%%	\item[Estatus:] %Terminado
%	\item[Objetivo:] Notificar al actor la omisión de algún dato obligatorio por ingresar.
%	\item[Redacción:] Campo obligatorio.
%	\item[Referenciado por:] \cdtIdRef{CU2}{Registrar paciente}, \cdtIdRef{CU4}{Editar información del paciente}
%\end{mensaje}
%
%%===========================  MSG3 ==================================
%\begin{mensaje}{MSG3}{Formato de campo incorrecto}{Error}
%	\item[Ubicación:] \msjCampo.
%%	\item[Estatus:] %Terminado
%	\item[Objetivo:] Indicar al actor que el dato ingresado en alguno de los campos del formulario no cumple con el tipo de dato definido en el diccionario de datos.
%	\item[Redacción:] El dato ingresado es incorrecto, favor de ingresar un dato válido.
%	\item[Referenciado por:] \cdtIdRef{CU2}{Registrar paciente}, \cdtIdRef{CU4}{Editar información del paciente}
%\end{mensaje}
%
%%===========================  MSG4 ==================================
%\begin{mensaje}{MSG4}{Número telefónico previamente registrado}{Error}
%	\item[Ubicación:] \msjCampo.
%	%	\item[Estatus:] %Terminado
%	\item[Objetivo:] Indicar al actor que el número telefónico ingresado ya ha sido registrado previamente.
%	\item[Redacción:] Número telefónico previamente registrado.
%	\item[Referenciado por:] \cdtIdRef{CU2}{Registrar paciente}, \cdtIdRef{CU4}{Editar información del paciente}
%\end{mensaje}
%
%
%%===========================  MSG5 ==================================
%\begin{mensaje}{MSG5}{Fecha de nacimiento inválida}{Error}
%	\item[Ubicación:] \msjCampo
%
%	\item[Objetivo:] Notificar al actor que la fecha que intenta registrar está fuera del rango considerado válido en el sistema.	
%	
%	\item[Redacción:] La fecha de nacimiento ingresada debe ser menor o igual a 150 años.
%	
%	\item[Referenciado por:] \cdtIdRef{CU2}{Registrar paciente}, \cdtIdRef{CU4}{Editar información del paciente}
%\end{mensaje}
%
%
%%===========================  MSG6 ==================================
%\begin{mensaje}{MSG6}{Eliminar Elemento}{Confirmación}
%	\item[Ubicación:] \msjEmergente.
%	\item[Objetivo:] Notificar al actor que está a punto de eliminar un elemento y que se necesita su aprobación para ello.
%	\item[Redacción:] ¿Desea eliminar DETERMINADO ELEMENTO VALOR?
%	\item[Parámetros:] El mensaje se muestra con base en los siguientes parámetros:
%	\begin{Citemize} 
%		\item DETERMINADO ELEMENTO: Es el elemento que se requiere eliminar.
%		\item VALOR: Es el valor asignado al atributo de la entidad, generalmente es la clave o el nombre.
%		%\item SELECCIÓN: Dependiendo del género en la oración se ocupa, seleccionado(masculino) ó seleccionada(femenino).
%	\end{Citemize}
%	\item[Ejemplo:] ¿Desea eliminar {\em al paciente Carlos Granados}?
%	\item[Referenciado por:] \cdtIdRef{CU5}{Eliminar paciente}
%
%\end{mensaje}
%
%%===========================  MSG ==================================