%SUPERFICIES FORESTALES

%----------------------------------------------------------
\subsection{Organización}
Las funcionalidades del la aplicación se encuentra organizadas por menús. Cada actor accede a un menú diferente dependiendo de su perfil, ya que este describe el ciclo de trabajo y las funciones que el actor puede realizar.


\hypertarget{menu:Usuario}{}	
\subsection{Menú Home: Usuario}
En la figura \ref{MN1} se muestra el menú mediante el cual el Usuario accederá a las funciones mas relevantes para el uso de la aplicación. Las opciones se listan a continuación:

\begin{itemize}
	\item \btnContactos [Gestión de Contactos] : Permite al usuario acceder a la gestión de contactos como agregar, editar, visualizar eliminar.
	\item \btnHome [Home] : Permite al usuario acceder a la pantalla principal de la aplicación donde podrá enviar notificaciones manuales.
	\item \btnVehiculos [Gestión Vehículos] : 
	\item \btnContactos [Gestión de Contactos] : Permite al usuario acceder a la gestión de vehículos como agregar, editar, visualizar eliminar.
\end{itemize}

\IUfig[.3]{../ModeloInteraccion/imagenes/menu/MNH}{MN1}{Menú home del usuario}

\pagebreak

\hypertarget{menu:UsuarioP}{}	
\subsection{Menú Principal: Usuario}
En la figura \ref{MN2} se muestra el menú mediante el usuario podrá acceder tanto a la información personal y de la cuanta, como a la configuración de notificaciones así como a la visualización de las mimas. Las opciones se listan a continuación:

\begin{itemize}
	\item Notificaciones: Permite al actor visualizar las notificaciones propias, como la de los contactos que tenga asociados.
	\item Configurar Notificaciones: Permite al actor modificar las notificaciones tanto automáticas como manuales que serán enviadas a los usuarios que el configure.
	\item Datos Personales: Permite al actor visualizar y mantener actualizada la información personal 
	\item Cuenta: Permite al actor visualizar la información general de la cuenta.
	\item Datos médicos: Permite al actor visualizar y mantener actualizada la información de sus datos médicos.
\end{itemize}

\IUfig[.3]{../ModeloInteraccion/imagenes/menu/MNP}{MN2}{Menú principal del usuario}

\pagebreak