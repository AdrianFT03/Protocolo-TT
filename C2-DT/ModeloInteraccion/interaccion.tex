%SUPERFICIES FORESTALES

%----------------------------------------------------------
\section{Entorno de trabajo}

    El entorno de trabajo es el medio por el cual el usuario interactúa con el sistema para poder acceder a la información registrada referente a los pacientes, así como a las mediciones de interés de temperatura y frecuencia. En este capítulo se describe el comportamiento y los elementos que conforman el entorno de 
    trabajo de la aplicación, como son: la disposición de los elementos principales y comunes de las pantallas, la iconografía, componentes, etc. \bigskip

    \begin{objetivos}
      \item Describir las áreas principales del entorno de trabajo.
      \item Describir la iconografía utilizada en las pantallas.
      \item Describir el mapa de navegación del sistema.
      \item Describir los componentes principales de las pantallas, tales como: controles de entrada, datos obligatorios, separadores, tablas de resultados, entre otros.
    \end{objetivos}
\\\\\\\\\\\\\\\\
%----------------------------------------------------------

\cfinput{ModeloInteraccion/interfaces}
\pagebreak
\cfinput{ModeloInteraccion/mensajes}