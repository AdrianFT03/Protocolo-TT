\subsection{CUA1.4 Cambiar contraseña}

\subsubsection{Objetivo}

% Explicar el objetivo para el que se construyo la interfaz, generalmente es la descripción de la actividad a desarrollar, como Seleccionar grupos para inscribir materias de un alumno, controlar el acceso al sistema mediante la solicitud de un login y password de los usuarios, etc.
	
Esta pantalla permite al usuario poder ingresa una nueva contraseña para actualizar la información de su cuenta.

\subsubsection{Diseño}

% Presente la figura de la interfaz y explique paso a paso ``a manera de manual de usuario'' como se debe utilizar la interfaz. No olvide detallar en la redacción los datos de entradas y salidas. Explique como utilizar cada botón y control de la pantalla, para que sirven y lo que hacen. Si el Botón lleva a otra pantalla, solo indique la pantalla y explique lo que pasará cuando se cierre dicha pantalla (la explicación sobre el funcionamiento de la otra pantalla estará en su archivo correspondiente).

    En la figura \ref{CUA1.4} se muestra la pantalla `` Cambiar contraseña ", por medio de la cual se mustra un formulario que servirá al usuario para cambiar la contraseña de su cuenta, esto con la finalidad de mantener aztualizada la información de la cuent.\\

   \IUfig[.4]{../ModeloComportamiento/Autenticacion/CUA1.1/images/IUA1Inicio}{CUA1.4}{Cambiar contraseña}

\subsubsection{Comandos}
    \begin{itemize}
        \item \cdtButton{¡LISTO!}: Permite al usuario actualizar la contraseña, dirige a la pantalla \cdtIdRef{IUA-1.1}{Iniciar Sesión}.
    	\item \btnRegresar[Regresar]: Permite al usuario salir de la pantalla, dirige a la pantalla \cdtIdRef{IUA-1.1}{Iniciar Sesión}.
    \end{itemize}

