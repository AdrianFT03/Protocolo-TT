\subsection{IUA-1.5 Finalizar Registro}

\subsubsection{Objetivo}

% Explicar el objetivo para el que se construyo la interfaz, generalmente es la descripción de la actividad a desarrollar, como Seleccionar grupos para inscribir materias de un alumno, controlar el acceso al sistema mediante la solicitud de un login y password de los usuarios, etc.
	
 Esta pantalla permite al actor informarle sobre la importancia del registro de los contactos de emergencia y sirve como punto de acceso para el registro de los mismos.

\subsubsection{Diseño}

% Presente la figura de la interfaz y explique paso a paso ``a manera de manual de usuario'' como se debe utilizar la interfaz. No olvide detallar en la redacción los datos de entradas y salidas. Explique como utilizar cada botón y control de la pantalla, para que sirven y lo que hacen. Si el Botón lleva a otra pantalla, solo indique la pantalla y explique lo que pasará cuando se cierre dicha pantalla (la explicación sobre el funcionamiento de la otra pantalla estará en su archivo correspondiente).

    En la figura \ref{IUA1.5C} se muestra la pantalla `` Finalizar Registro ", por medio de la cual se le informa al usuario la importancia de que registre los contactos que estarán disponibles en su cuenta de In-Help para el envio de notificaciones, así como la configuración de los mismos.

    

   \IUfig[.4]{../ModeloComportamiento/Autenticacion/CUA1.7/images/IUA1_5C}{IUA1.5C}{Finalizar Registro}
   

\subsubsection{Comandos}
    \begin{itemize}
    	\item \cdtButton{SIGUIENTE}: Permimte al actor guardar la información de contactos, dirige a la pantalla \cdtIdRef{IUA1.6}{Registro de Contactos}.
        \item OMITIR: Permite al omitir la sección, dirige a la pantalla \cdtIdRef{IUA1.7}{Registro Finalizado}.
        \item \btnRegresar[Regresar]: Permite al actor salir de la palicación, dirige a la pantalla \cdtIdRef{IUA1.5A}{Finalizar Registro}.
    \end{itemize}