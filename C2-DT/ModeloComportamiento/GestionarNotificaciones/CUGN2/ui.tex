\subsection{IUGN2  Información de Notificación}

\subsubsection{Objetivo}

% Explicar el objetivo para el que se construyo la interfaz, generalmente es la descripción de la actividad a desarrollar, como Seleccionar grupos para inscribir materias de un alumno, controlar el acceso al sistema mediante la solicitud de un login y password de los usuarios, etc.
	
    Esta pantalla permite al actor visualizar la información general de una notificación de un evento que se registro en In-Help.

\subsubsection{Diseño}

% Presente la figura de la interfaz y explique paso a paso ``a manera de manual de usuario'' como se debe utilizar la interfaz. No olvide detallar en la redacción los datos de entradas y salidas. Explique como utilizar cada botón y control de la pantalla, para que sirven y lo que hacen. Si el Botón lleva a otra pantalla, solo indique la pantalla y explique lo que pasará cuando se cierre dicha pantalla (la explicación sobre el funcionamiento de la otra pantalla estará en su archivo correspondiente).

    En la figura \ref{IUGN2} se muestra la pantalla `` Información de Notificación ", por medio de la cual se muestra la información mas relevante del registro de un evento para el actor.\\
    En la figura \ref{IUGN2A} se muestra la pantalla `` Información de Notificación ", por medio de la cual se muestra la información mas relevante del registro de un evento para el contacto del actor.\\

   \IUfig[.4]{../ModeloComportamiento/GestionarNotificaciones/CUGN2/images/IUGN2}{IUGN2}{Información de Notificación}
   \IUfig[.4]{../ModeloComportamiento/GestionarNotificaciones/CUGN2/images/IUGN2}{IUGN2A}{Información de Notificación}

\subsubsection{Comandos}
    \begin{itemize}
    	\item \btnRegresar[Regresar]. Permite al actor salir de la pantalla, dirige a la pantalla \cdtIdRef{IUGN1}{Gestión de Notificaciones}
    	\item \btnDetalles[Detalles]. Permite al actor visualizar mayor información del evento, dirige a la pantalla \cdtIdRef{IUGN3}{Información General}
    \end{itemize}

