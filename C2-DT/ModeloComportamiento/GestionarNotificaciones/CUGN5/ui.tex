\subsection{IUGN5 Configurar Notificaciones}

\subsubsection{Objetivo}

% Explicar el objetivo para el que se construyo la interfaz, generalmente es la descripción de la actividad a desarrollar, como Seleccionar grupos para inscribir materias de un alumno, controlar el acceso al sistema mediante la solicitud de un login y password de los usuarios, etc.
	
    Esta pantalla permite al actor acceder a la configuración de notificaciones.

\subsubsection{Diseño}

% Presente la figura de la interfaz y explique paso a paso ``a manera de manual de usuario'' como se debe utilizar la interfaz. No olvide detallar en la redacción los datos de entradas y salidas. Explique como utilizar cada botón y control de la pantalla, para que sirven y lo que hacen. Si el Botón lleva a otra pantalla, solo indique la pantalla y explique lo que pasará cuando se cierre dicha pantalla (la explicación sobre el funcionamiento de la otra pantalla estará en su archivo correspondiente).

    En la figura \ref{IUGN5} se muestra la pantalla `` Configurar Notificaciones ", por medio de la cual se muestra una lista donde el usuario podrá visualizar los coches que ha registrado para así poder configurar las notificaciones a cada uno de ellos.\\

   \IUfig[.4]{../ModeloComportamiento/GestionarNotificaciones/CUGN5/images/IUGN5}{IUGN5}{Configurar Notificaciones}
\subsubsection{Comandos}
    \begin{itemize}
    	\item \btnRegresar[Regresar]: Permite al usuario salir de la pantalla, dirige a la pantalla principal.
    \end{itemize}

