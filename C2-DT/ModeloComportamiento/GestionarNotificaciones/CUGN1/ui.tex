\subsection{IUGC1 Gestión de Notificaciones}

\subsubsection{Objetivo}

% Explicar el objetivo para el que se construyo la interfaz, generalmente es la descripción de la actividad a desarrollar, como Seleccionar grupos para inscribir materias de un alumno, controlar el acceso al sistema mediante la solicitud de un login y password de los usuarios, etc.
	
    Esta pantalla permite al usuario visualizar  las notificaciones que se tienen registradas, las cuales fueron generadas en una determinada fecha, ya sea personales o de contactos asociados.

\subsubsection{Diseño}

% Presente la figura de la interfaz y explique paso a paso ``a manera de manual de usuario'' como se debe utilizar la interfaz. No olvide detallar en la redacción los datos de entradas y salidas. Explique como utilizar cada botón y control de la pantalla, para que sirven y lo que hacen. Si el Botón lleva a otra pantalla, solo indique la pantalla y explique lo que pasará cuando se cierre dicha pantalla (la explicación sobre el funcionamiento de la otra pantalla estará en su archivo correspondiente).

    En la figura \ref{IUGN1} se muestra la pantalla `` Gestión de Notificaciones ", por medio de la cual se muestran una lista de registros, los cuales corresponden a las notificaciones que se tienen asociadas al usuario ya sean personales o de contactos.\\
    En la figura \ref{IUGN1A} se muestra la pantalla `` Gestión de Notificaciones ", por medio de la cual se muestran una lista de registros, los cuales corresponden a las notificaciones de usuarios que tienen relación con el actor.\\

   \IUfig[.4]{../ModeloComportamiento/GestionarNotificaciones/CUGN1/images/IUGN1}{IUGN1}{Gestión de Notificaciones}
	\IUfig[.4]{../ModeloComportamiento/GestionarNotificaciones/CUGN1/images/IUGN1A}{IUGN1A}{Gestión de Notificaciones}
\subsubsection{Comandos}
    \begin{itemize}
    	\item \btnDetalle[Detalle]: Permite visualizar la información de la notificación, dirige a la pantalla \cdtIdRef{IUGN2}{Infomración de notificación}
    \end{itemize}

