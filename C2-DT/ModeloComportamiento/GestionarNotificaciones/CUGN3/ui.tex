\subsection{IUGC3 Información General}

\subsubsection{Objetivo}

% Explicar el objetivo para el que se construyo la interfaz, generalmente es la descripción de la actividad a desarrollar, como Seleccionar grupos para inscribir materias de un alumno, controlar el acceso al sistema mediante la solicitud de un login y password de los usuarios, etc.
	
    Esta pantalla permite al actor visualizar la información general tanto del vehículo como del dispositivo de una notificación de un evento que se registro en In-Help.

\subsubsection{Diseño}

% Presente la figura de la interfaz y explique paso a paso ``a manera de manual de usuario'' como se debe utilizar la interfaz. No olvide detallar en la redacción los datos de entradas y salidas. Explique como utilizar cada botón y control de la pantalla, para que sirven y lo que hacen. Si el Botón lleva a otra pantalla, solo indique la pantalla y explique lo que pasará cuando se cierre dicha pantalla (la explicación sobre el funcionamiento de la otra pantalla estará en su archivo correspondiente).

    En la figura \ref{IUGN3} se muestra la pantalla `` Información General ", por medio de la cual se muestra la información mas relevante del vehículo así como del dispositivo en el registro de un evento.\\

   \IUfig[.4]{../ModeloComportamiento/GestionarNotificaciones/CUGN3/images/IUGN3}{IUGN3}{Información General}

\subsubsection{Comandos}
    \begin{itemize}
    	\item \btnRegresar[Regresar]. Permite al actor salir de la pantalla, dirige a la pantalla \cdtIdRef{IUGN2}{Información de Notificación}.
    \end{itemize}

