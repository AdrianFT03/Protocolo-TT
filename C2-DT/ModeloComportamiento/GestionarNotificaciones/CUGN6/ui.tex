\subsection{IUGN6 Configurar Notificación}

\subsubsection{Objetivo}

% Explicar el objetivo para el que se construyo la interfaz, generalmente es la descripción de la actividad a desarrollar, como Seleccionar grupos para inscribir materias de un alumno, controlar el acceso al sistema mediante la solicitud de un login y password de los usuarios, etc.
	
    Esta pantalla permite al actor acceder a las acciones para la configuración de notificaciones de un vehículo en especifico.

\subsubsection{Diseño}

% Presente la figura de la interfaz y explique paso a paso ``a manera de manual de usuario'' como se debe utilizar la interfaz. No olvide detallar en la redacción los datos de entradas y salidas. Explique como utilizar cada botón y control de la pantalla, para que sirven y lo que hacen. Si el Botón lleva a otra pantalla, solo indique la pantalla y explique lo que pasará cuando se cierre dicha pantalla (la explicación sobre el funcionamiento de la otra pantalla estará en su archivo correspondiente).

    En la figura \ref{IUGN6} se muestra la pantalla `` Configurar Notificación ", en la cual se muestra una lista con todos los contactos que tiene registrado el actor en In-Help, así como la información del coche que selecciono.\\

   \IUfig[.4]{../ModeloComportamiento/GestionarNotificaciones/CUGN6/images/IUGN6}{IUGN6}{Configurar Notificación}
\subsubsection{Comandos}
    \begin{itemize}
    	\item \btnRegresar[Regresar]: Permite al actor salir, dirige a la pantalla \cdtIdRef{Configurar Notificaciones}.
    	\item \btnOn[Activar/Desactivar]: Permite al actor activar a un contacto, dirige a la misma pantalla.
    	\item \btnGestionar[Configurar]: Permite al actor acceder a la configuración de la información de la notificación, dirige a la pantalla \cdtIdRef{IUGN8}{Configurar Notificación} 
    \end{itemize}

