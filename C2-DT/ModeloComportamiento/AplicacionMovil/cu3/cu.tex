\begin{UseCase}{CU3}{Consultar información del paciente}
	{	
		Este caso de uso permite al \hyperlink{actor:usuario}{Usuario} consultar la información de un paciente en específico, con el fin de obtener un informe general sobre las todas las mediciones de temperatura corporal y frecuencia cardíaca realizadas.
		 
	}
\UCccitem{Versión}{0.1}
\UCccsection{Administración de Requerimientos}
\UCccitem{Autor}{María Elsi Bernabé Aparicio}
%\UCccitem{Evaluador}{}
\UCccitem{Operación}{Registro}
\UCccitem{Prioridad}{Alta}
\UCccitem{Complejidad}{Media}
\UCccitem{Volatilidad}{Alta}
\UCccitem{Madurez}{Alta}
\UCccitem{Estatus}{Edición}
\UCccitem{Fecha del último estatus}{18 de octubre de 2018}

% Copie y pegue este bloque tantas veces como revisiones tenga el caso de uso.
% Esta sección la debe llenar solo el Revisor
%--------------------------------------------------------
%\UCccsection{Revisión Versión 0.1} % Anote la versión que se revisó.
%\UCccitem{Fecha}{}
%\UCccitem{Evaluador}{Elsi Bernabé Aparicio}
%\UCccitem{Resultado}{}
%\UCccitem{Observaciones}{}
%-------------------------------------------------------------------
	\UCsection{Atributos}
	\UCitem{Actor(es)}{\hyperlink{actor:usuario}{Usuario}.}
	\UCitem{Propósito}{Proporcionar un mecanismo que le permita al \hyperlink{actor:usuario}{Usuario} consultar la información general del paciente seleccionado.}
	\UCitem{Entradas}{
		\begin{UClist}
			\UCli Paciente del cuál se requiere consultar su información.
		\end{UClist}	
	}
	\UCitem{Salidas}{
		\begin{UClist}
			\UCli Nombre del paciente seleccionado. \ioObtener
			\UCli Edad calculada del paciente. \ioCalcular
			\UCli Sexo del paciente. \ioObtener
			\UCli Última medición registrada de temperatura corporal del paciente. \ioObtener
			\UCli Última medición registrada de frecuencia cardíaca del paciente. \ioObtener
			\UCli Promedio de todas las mediciones de temperatura corporal registradas para el paciente. \ioCalcular
			\UCli Promedio de todas las mediciones de frecuencia cardíaca registradas para el paciente- \ioCalcular
		\end{UClist}
		
	}
	\UCitem{Precondiciones}{
		\begin{UClist}
			\UCli Ninguna.
		\end{UClist}
	}
	
	\UCitem{Postcondiciones}{
		\begin{UClist}
			\UCli Ninguna.
		\end{UClist}
	}

	\UCitem{Reglas de negocio}{
		\begin{UClist}
			\UCli \cdtIdRef{RN4}{Cálculo del promedio de mediciones de temperatura corporal}: Calcula el promedio de mediciones de temperatura corporal para el paciente seleccionado.
			\UCli \cdtIdRef{RN5}{Cálculo del promedio de mediciones de frecuencia cardíaca}: Calcula el promedio de mediciones de frecuencia cardíaca para el paciente seleccionado.
		\end{UClist}
	}
	
	\UCitem{Errores}{
		\begin{UClist}
			\UCli Ninguno.
		\end{UClist}
	}
	\UCitem{Tipo}{Secundario, extiende del caso de uso \cdtIdRef{CU1}{Consultar pacientes}.}
\end{UseCase}

\begin{UCtrayectoria}
	\UCpaso[\UCactor] Selecciona el nombre del paciente del cuál requiere consultar su información en la pantalla \cdtIdRef{IU1}{Consultar pacientes}.
	\UCpaso[\UCsist] Obtiene el nombre del paciente seleccionado.
	\UCpaso[\UCsist] Obtiene el sexo del paciente.
	\UCpaso[\UCsist] Calcula la edad del paciente con base en  su fecha de nacimiento.
	\UCpaso[\UCsist] Obtiene el último valor medido de temperatura corporal.
	\UCpaso[\UCsist] Obtiene el último valor calculado de frecuencia cardíaca.
	\UCpaso[\UCsist] Calcula el promedio de todas las mediciones de temperatura corporal registradas como lo especifica la regla de negocio \cdtIdRef{RN4}{Cálculo del promedio de mediciones de temperatura corporal}.
	\UCpaso[\UCsist] Calcula el promedio de todas las mediciones de frecuencia cardíaca registradas como lo especifica la regla de negocio \cdtIdRef{RN5}{Cálculo del promedio de mediciones de frecuencia cardíaca}.
	\UCpaso[\UCsist] Muestra la pantalla \cdtIdRef{IU3}{Consultar información del paciente} con la información obtenida y los datos calculados del paciente.
	\UCpaso[\UCactor] \label{cu3:extension}Controla las acciones posibles de realizar para el paciente mediante los iconos \btnMonitoreo{}(Consultar registros de signos vitales), \btnEditar{} (Editar información del paciente) y \btnEliminar{} (Eliminar información del paciente).
\end{UCtrayectoria}


\subsection{Puntos de extensión}
\UCExtensionPoint
{El actor requiere actualizar la los datos registrados previamente para el paciente}
{Paso \ref{cu3:extension} de la trayectoria principal}
{\cdtIdRef{CU4}{Editar información del paciente}}

\UCExtensionPoint
{El actor requiere eliminar al paciente}
{Paso \ref{cu3:extension} de la trayectoria principal}
{\cdtIdRef{CU5}{Eliminar paciente}}

\UCExtensionPoint
{El actor requiere consultar el historial de mediciones de temperatura corporal y frecuencia cardíaca}
{Paso \ref{cu3:extension} de la trayectoria principal}
{\cdtIdRef{CU6}{Consultar registros de signos vitales}}

\UCExtensionPoint
{El actor requiere solicitar una nueva medición de temperatura corporal y frecuencia cardíaca del paciente}
{Paso \ref{cu3:extension} de la trayectoria principal}
{\cdtIdRef{CU7}{Solicitar medición}}