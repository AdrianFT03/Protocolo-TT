\subsection{IU3 Consultar información del paciente}

\subsubsection{Objetivo}
	
Esta pantalla permite al \hyperlink{actor:usuario}{Usuario} consultar la información general del paciente seleccionado.

\subsubsection{Diseño}
En la figura \ref{IU3} se muestra la pantalla ''Consultar información del paciente'', por medio de la cual el usuario podrá
visualizar, editar o eliminar la información de un paciente.\\

La pantalla se divide en tres secciones, las cuales se describen a continuación:

\begin{enumerate}
	\item \textbf{Información general:} Muestra la información general de un paciente, contiene algunos de los datos ingresados en el momento de su registro como el sexo y la edad que es calculada con base en la fecha de nacimiento del paciente.
	
	\item \textbf{Últimas mediciones registradas:} Muestra las últimas mediciones de temperatura corporal y frecuencia cardíaca que fueron medidas y enviadas a través de un mensaje SMS.
	
	\item \textbf{Promedio de mediciones:} Indica el promedio de temperatura corporal y frecuencia cardíaca de un paciente con base en todas las mediciones registradas para un paciente, como lo indican las reglas de negocio \cdtIdRef{RN4}{Cálculo del promedio de mediciones de temperatura corporal} y \cdtIdRef{RN5}{Cálculo del promedio de mediciones de frecuencia cardíaca}.
\end{enumerate}
    
 En la parte inferior de la pantalla, se muestra la opción \btnMonitoreo{}, la cual es el acceso a la pantalla en donde el usuario podrá visualizar el historial de registro de mediciones de un paciente.
 
 \IUfig[.6]{../ModeloComportamiento/AplicacionMovil/cu3/iu3.png}{IU3}{Consultar información del paciente}

\subsubsection{Comandos}
	\begin{itemize}
		\item \btnEditar{} [Editar información del paciente]: Permite al actor actualizar los datos registrado para un paciente. Dirige a la pantalla \cdtIdRef{IU4}{Editar Información del paciente}.
		\item \btnEliminar{} [Eliminar paciente]: Permite al actor eliminar los registros e información de un paciente en específico. Muestra el mensaje emergente \cdtIdRef{MSG6}{Eliminar elemento}.
		\item \btnMonitoreo [Consultar registros]: Permite al actor visualizar el registro de los signos vitales del paciente. Dirige a la pantalla \cdtIdRef{IU6}{Consultar registros de signos vitales}.
	\end{itemize}
\clearpage