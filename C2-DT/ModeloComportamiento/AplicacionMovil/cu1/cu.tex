\begin{UseCase}{CU1}{Consultar pacientes}
	{	
		Permite al \hyperlink{actor:usuario}{Usuario} visualizar el nombre de todos los pacientes que ha registrado, con el fin de acceder a la funcionalidades específicas de cada uno de ellos. 
	}
\UCccitem{Versión}{0.1}
\UCccsection{Administración de Requerimientos}
\UCccitem{Autor}{María Elsi Bernabé Aparicio}
%\UCccitem{Evaluador}{}
\UCccitem{Operación}{Consultar}
\UCccitem{Prioridad}{Alta}
\UCccitem{Complejidad}{Media}
\UCccitem{Volatilidad}{Alta}
\UCccitem{Madurez}{Alta}
\UCccitem{Estatus}{Edición}
\UCccitem{Fecha del último estatus}{18 de octubre de 2018}

% Copie y pegue este bloque tantas veces como revisiones tenga el caso de uso.
% Esta sección la debe llenar solo el Revisor
%--------------------------------------------------------
%\UCccsection{Revisión Versión 0.1} % Anote la versión que se revisó.
%\UCccitem{Fecha}{}
%\UCccitem{Evaluador}{Elsi Bernabé Aparicio}
%\UCccitem{Resultado}{}
%\UCccitem{Observaciones}{}
%-------------------------------------------------------------------
	\UCsection{Atributos}
	\UCitem{Actor(es)}{\hyperlink{actor:usuario}{Usuario}.}
	\UCitem{Propósito}{Proporcionar un mecanismo que le permita al \hyperlink{actor:usuario}{Usuario} visualizar a todos los pacientes registrados en la aplicación móvil con el fin de controlar las distintas acciones que se puedan aplicar a un paciente.}
	\UCitem{Entradas}{
		\begin{UClist}
			\UCli Ninguna.
		\end{UClist}	
	}
	\UCitem{Salidas}{
		\begin{UClist}
			\UCli Nombre da cada uno de los pacientes registrados.
		\end{UClist}
		
	}
	\UCitem{Precondiciones}{
		\begin{UClist}
			\UCli Ninguna.
		\end{UClist}
	}
	
	\UCitem{Postcondiciones}{
		\begin{UClist}
			\UCli Ninguna.
		\end{UClist}
	}

	\UCitem{Reglas de negocio}{
		\begin{UClist}
			\UCli Ninguna.
		\end{UClist}
	}
	
	\UCitem{Errores}{
		\begin{UClist}
			\UCli Ninguno.
		\end{UClist}
	}
	\UCitem{Tipo}{Primario}
\end{UseCase}

\begin{UCtrayectoria}
	\UCpaso[\UCactor] Ingresa a la aplicación desde su dispositivo móvil.
	\UCpaso[\UCsist] Obtiene el nombre de los pacientes registrados.
	\UCpaso[\UCsist] Muestra la pantalla \cdtIdRef{IU1}{Consultar pacientes} con una lista de todos los pacientes registrados.
	\UCpaso[\UCactor] \label{cu1:extension}Visualiza la lista de pacientes y controla las acciones posibles de realizar. \refTray{A}	
\end{UCtrayectoria}


%-----------------------------------------------------------
\begin{UCtrayectoriaA}[Fin del caso de uso]{A}{El actor requiere buscar un paciente de la lista.}
	\UCpaso[\UCactor] Ingresa el nombre del paciente.
	\UCpaso[\UCsist] Obtiene las coincidencias de la información ingresada con el nombre de los pacientes registrados.
	\UCpaso[\UCsist] Muestra el nombre de todos los pacientes que coinciden con la búsqueda.
\end{UCtrayectoriaA}





\subsection{Puntos de extensión}
\UCExtensionPoint
{El actor consultar la información de un paciente en específico.}
{Paso \ref{cu1:extension} de la trayectoria principal}
{\cdtIdRef{CU3}{Consultar información del paciente}}