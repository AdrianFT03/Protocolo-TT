\subsection{IU2 Registrar paciente}

\subsubsection{Objetivo}
	
Esta pantalla permite al \hyperlink{actor:usuario}{Usuario} registrar un nuevo paciente del cual requiere monitorear los signos vitales de temperatura y frecuencia cardíaca.

\subsubsection{Diseño}
En la figura \ref{IU2} se muestra la pantalla ''Registrar paciente'', por medio de la cual el usuario podrá ingresar los datos personales y número de teléfono de un paciente que requiera registrar para el monitoreo de sus signos vitales.\\

Los datos necesarios para el registro de un paciente son:
\begin{enumerate}
	\item Nombre del paciente que será monitoreado.
	\item Número telefónico de la tarjeta SIM que tendrá integrada el módulo de comunicación GSM para en envío de mensajes.
	\item Fecha de nacimiento del paciente, la cual debe encontrarse dentro del rango establecido con base en la regla de negocio \cdtIdRef{RN3}{Fecha de nacimiento válida}.
	\item Sexo del paciente.
\end{enumerate}

Una vez validados todos los datos del paciente, éste será registrado y podrá ser visualizado en la lista de pacientes registrados de la pantalla \cdtIdRef{IU1}{Consultar pacientes}.

    
 \IUfig[.6]{../ModeloComportamiento/AplicacionMovil/cu2/iu2.png}{IU2}{Registrar paciente}

\subsubsection{Comandos}
	\begin{itemize}
		\item \btnRegistrar{} [Registrar paciente]: Permite al actor confirmar los datos ingresados del paciente para finalizar su registro. Dirige a la pantalla \cdtIdRef{IU1}{Consultar pacientes}.
	\end{itemize}
\clearpage