\subsection{IUGC1 Gestión de Contactos}

\subsubsection{Objetivo}

% Explicar el objetivo para el que se construyo la interfaz, generalmente es la descripción de la actividad a desarrollar, como Seleccionar grupos para inscribir materias de un alumno, controlar el acceso al sistema mediante la solicitud de un login y password de los usuarios, etc.
	
    Esta pantalla permite al usuario visualizar las acciones principales que puede realizar en el módulo de contactos, brindandole acceso a las acciones correspondientes.

\subsubsection{Diseño}

% Presente la figura de la interfaz y explique paso a paso ``a manera de manual de usuario'' como se debe utilizar la interfaz. No olvide detallar en la redacción los datos de entradas y salidas. Explique como utilizar cada botón y control de la pantalla, para que sirven y lo que hacen. Si el Botón lleva a otra pantalla, solo indique la pantalla y explique lo que pasará cuando se cierre dicha pantalla (la explicación sobre el funcionamiento de la otra pantalla estará en su archivo correspondiente).

    En la figura \ref{IUGC1} se muestra la pantalla `` Gestión de Contactos ", por medio de la cual se muestran dos acciones, las cuales le permiten al usuario acceder a las operaciones que puede relizar para la festión de contactos.\\

   \IUfig[.4]{../ModeloComportamiento/Autenticacion/CUA1.1/images/IUA-1_1Login}{IUGC1}{Gestión de Contactos}

\subsubsection{Comandos}
    \begin{itemize}
    	\item \btnRContacto[Contactos Registrados]: Permite al usuario acceder a la gestión de contactos registrados, dirige a la pantalla \cdtIdRef{CUGC1.1}{Contactos Registrados}.
        \item \btnAContacto[Agregar Contactos]: Permite al usuario agregar contactos, dirige a la pantalla \cdtIdRef{IUGC1.3}{Registrar Contacto}.
    \end{itemize}

