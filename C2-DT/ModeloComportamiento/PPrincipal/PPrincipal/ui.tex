\subsection{IUPP1A Pantalla Principal}

\subsubsection{Objetivo}

% Explicar el objetivo para el que se construyo la interfaz, generalmente es la descripción de la actividad a desarrollar, como Seleccionar grupos para inscribir materias de un alumno, controlar el acceso al sistema mediante la solicitud de un login y password de los usuarios, etc.
	
    Esta pantalla permite al usuario acceder a todas las acciones que estarán disponibles para el usuario en In-Help, ya que esta es la pantalla principal de la aplicación.

\subsubsection{Diseño}

% Presente la figura de la interfaz y explique paso a paso ``a manera de manual de usuario'' como se debe utilizar la interfaz. No olvide detallar en la redacción los datos de entradas y salidas. Explique como utilizar cada botón y control de la pantalla, para que sirven y lo que hacen. Si el Botón lleva a otra pantalla, solo indique la pantalla y explique lo que pasará cuando se cierre dicha pantalla (la explicación sobre el funcionamiento de la otra pantalla estará en su archivo correspondiente).

    En la figura \ref{IUPP1A} se muestra la pantalla `` Pantalla Principal ", por medio de la cual el usuario podrá acceder al envío de notificaciones, gestión de vehículos y gestión de contactos, esto desde los iconos correspondientes. De igual forma en la pantalla se muestra el acceso al menú del usuario el cual esta representado por el icono \btnMenu.
    

   \IUfig[.4]{../ModeloComportamiento/PPrincipal/PPrincipal/images/PP1}{IUPP1A}{Pantalla Principal}

\subsubsection{Comandos}
    \begin{itemize}
    	
    		\item \btnMenu [Menú]: Permite al usuario visualizar el menú del usuario, dirige a la misma pantalla.
    		\item \btnHome [Home]: Permite al usuario visualizar la pantalla principal del home, dirige a la pantalla \cdtIdRef{IUPP1A}{Pantalla Principal}
    		\item \btnVehiculos [Vehículos]: Permite al usuario acceder a la gestión de vehículos, dirige a la pantalla 
    		\item \btnContactos [Contactos]: Permite al usuario acceder a la gestión de contactos, dirige a la pantalla 
    		\item \btnNotifiacion [Notificación manual]: Permite al usuario enviar notificación manual, dirige a la misma pantalla.
    	
    \end{itemize}

%\subsubsection{Mensajes}
%
%    \begin{description}
%		\item[\cdtIdRef{MSG4}{Datos inválidos}] Se muestra este mensaje de error cuando alguno de los datos marcados como requeridos (con asteriscos rojos) no es proporcionado por el usuario, señalando el dato que hizo falta.
%    \end{description}
