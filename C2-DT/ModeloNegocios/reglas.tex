% El tipo de regla de negocio (tercer parámetro del entorno 'BusinessRule') se describe en la siguiente tabla:
%---------------------------------------------------------------------------------------------------------------,
% TIPOS		|		DEFINICION		|	EJEMPLO						|
%---------------|---------------------------------------|-------------------------------------------------------|
% Habilitador   | La sentencia habilita o restringe 	| * Se pueden recibir solicitudes del tipo A, B y C.	|
%		| hacer algo o  una funcionalidad.	| * Se permite hacer algo si se tiene el estado X.	|
%---------------|---------------------------------------|-------------------------------------------------------|
% Cronometrado	| Se permite de manera controlada 	| * Se permiten hasta dos solicitudes del tipo D	| 
%		| por un contador.			|   por persona.					|
% 		|					| * El acceso al sistema se permite si no se tiene 	|
%  		|					|   más de X número de intentos fallidos.		|
%---------------|---------------------------------------|-------------------------------------------------------|
% Ejecutivo	| Autorizado por un superior, un perfil | * Se permite registrar extemporaneamente si lo 	|
%		| particular debe autorizar.		|   autoriza X.						|
%---------------|---------------------------------------|-------------------------------------------------------|
% Derivación	| Son de cálculo e inferencia, 		| * Un alumno irregular es aquel que tiene las 		|
%		| es un cálculo o conclusión derivados 	|   siguientes cacteristicas: A, B, C. 			|
%		| de un conjunto de datos. Puede ser una| * El formato de un correo o CURP.			|
%		| fórmula que dice cómo calcular algo 	|							|
%		| o el formato de un dato.		|							|
%---------------|---------------------------------------|-------------------------------------------------------|
% Restricción	| Restringe una funcionalidad o relación| * Traslape de fechas o periodos empalmados.		|
% 		| entre dos o mas objetos.		|							|
%  		|					|							|
%---------------------------------------------------------------------------------------------------------------'

% No editar las reglas cuyo estatus es APROBADO.

\section{Reglas de negocio}
%%------------------------------------------------------------------------------------------------------------------
%============================== RN1 =================================
\begin{BusinessRule}{RN1}{Campos obligatorios}
	{Restricción}
	{Controla la operación}
	\BRitem{Versión}{1.0}
	%\BRitem{Autor}{}
	\BRitem{Estatus}{Revisión}
	\BRitem{Descripción}{La información que se proporcione en los campos marcados como obligatorios debe ser ingresada para poder continuar con la operación requerida.}
	\BRitem{Referenciado por}{
		\cdtIdRef{CU2}{Registrar paciente}, \cdtIdRef{CU4}{Editar información del paciente}
	}
	
\end{BusinessRule}
%------------------------------------------------------------------------------------------------------------------

%%============================== RN2 =================================
%\begin{BusinessRule}{RN2}{Información correcta}
%	{Restricción}
%	{Controla la operación}
%	\BRitem{Versión}{1.0}
%	\BRitem{Estatus}{Revisión}
%	\BRitem{Descripción}{Todos los datos proporcionados al sistema deben pertenecer al tipo de dato especificado y respetar el formato con base en lo definido en la entidad del diccionario de datos.}
%	\BRitem{Referenciado por}{ 
%		\cdtIdRef{CU2}{Registrar paciente}, \cdtIdRef{CU4}{Editar información del paciente}
%	}
%\end{BusinessRule}
%
%%------------------------------------------------------------------------------------------------------------------
%
%%============================== RN3 =================================
%\begin{BusinessRule}{RN3}{Fecha de nacimiento válida}
%	{Restricción}
%	{Controla la operación}
%	\BRitem{Versión}{1.0}
%	\BRitem{Estatus}{Revisión}
%	\BRitem{Descripción}{Para que una fecha de nacimiento registrada en la aplicación móvil se considere válida, se debe cumplir que.
%		
%	\begin{center}
%		$F_{nac} \geq (F_{act}-150$ años) 
%	\end{center}	
%		
%	}
%	\BRitem{Referenciado por}{ 
%		\cdtIdRef{CU2}{Registrar paciente}, \cdtIdRef{CU4}{Editar información del paciente}
%	}
%\end{BusinessRule}
%
%
%%============================== RN4 =================================
%\begin{BusinessRule}{RN4}{Cálculo del promedio de mediciones de temperatura corporal}
%	{Derivación}
%	{Controla la operación}
%	\BRitem{Versión}{0.1}
%	\BRitem{Estatus}{Revisión}
%	\BRitem{Descripción}{
%		El cálculo del promedio de temperatura corporal de un paciente se realizará tomando todas las mediciones de temperatura registradas y dividiéndolo entre el número total de ellas.
%	}
%	
%	\BRitem{Sentencia}{El cálculo del promedio de la temperatura corporal de un paciente se realiza de la siguiente forma:\\
%
%		\begin{center}
%			$P_{temp} = \frac{T_{1}\ +\ T_{2}\ +\ T_{3}\ +\ ...\ +\ T_{n}}{N}$	
%		\end{center}
%	
%		
%		Donde: 
%		
%		$P_{temp} =$ Promedio de la temperatura corporal del paciente. \\
%		$T_{n} =$ Medición n de temperatura corporal del paciente. \\
%		$N =$ Total de mediciones de temperatura registradas para el paciente. \\
%	}
%	
%	\BRitem{Referenciado por}{\cdtIdRef{CU3}{Consultar información del paciente}}
%	
%\end{BusinessRule}
%
%%============================== RN5 =================================
%\begin{BusinessRule}{RN5}{Cálculo del promedio de mediciones de frecuencia cardíaca}
%	{Derivación}
%	{Controla la operación}
%	\BRitem{Versión}{0.1}
%	\BRitem{Estatus}{Revisión}
%	\BRitem{Descripción}{
%		El cálculo del promedio de frecuencia cardíaca de un paciente se realizará tomando todas las mediciones de frecuencia cardíaca registradas y dividiéndolo entre el número total de ellas.
%	}
%	
%	\BRitem{Sentencia}{El cálculo del promedio de la frecuencia cardíaca de un paciente se realiza de la siguiente forma:\\
%		
%		\begin{center}
%			$P_{fc} = \frac{FC_{1}\ +\ FC_{2}\ +\ FC_{3}\ +\ ...\ +\ FC_{n}}{N}$	
%		\end{center}
%		
%		
%		Donde: 
%		
%		$P_{fc} =$ Promedio de la frecuencia cardíaca del paciente. \\
%		$FC_{n} =$ Medición n de frecuencia cardíaca del paciente. \\
%		$N =$ Total de mediciones de frecuencia cardíaca registradas para el paciente. \\
%	}
%	
%	\BRitem{Referenciado por}{\cdtIdRef{CU3}{Consultar información del paciente}}
%	
%\end{BusinessRule}