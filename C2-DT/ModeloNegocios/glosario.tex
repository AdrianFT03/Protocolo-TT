\label{sec:glosario}
    Este capítulo describe los términos usados a lo largo del documento que tienen un significado singular en la App In-Help, los cuales se considera necesario definirlos para evitar ambigüedades o malos entendidos.
    La lista de términos se encuentra agrupada por áreas de conocimiento:
\begin{Citemize}
    \item Términos técnicos: Agrupa los términos que tienen que ver con el sistema.
    \item Términos del negocio: Agrupa los términos que tienen significado dentro de la SMAGEM.
\end{Citemize}

    Para fines de este documento la siguiente lista de términos se debe interpretar como se describen en este capítulo.

%  \RCitem{ARM-01}{\TOCHK{En todos los términos, cuando menciones ``tipo de dato'' que sea una referencia al término.}}{31 de julio de 2014}
%  \RCitem{ARM-02}{\TOCHK{Agrega el término ``Georreferenciación'' en el documento de ANPs ya está la definición :)}}{31 de julio de 2014}
%  \RCitem{ARM-03}{\TOCHK{El término ``Tipo de documento legal'' ampliar su definición}}{31 de julio de 2014}
   
%====================================================================
\section{Términos técnicos}
\label{gls:terminosTecnicos}

  En esta sección se definen los términos técnicos que se utilizan para describir el comportamiento del sistema.
  
  \begin{description}
  
    \BRterm{gls:alfanumerico}{Alfanumérico:} Es un \cdtRef{gls:tipoDato}{tipo de dato} definido por el conjunto de caracteres numéricos y alfabéticos.
   
    
    \BRterm{gls:atributo}{Atributo:} Son las características que definen o identifican a una entidad en un conjunto de entidades.

    \BRterm{gls:booleano}{Booleano:} Es un \cdtRef{gls:tipoDato}{tipo de dato} que puede tomar los siguientes valores: verdadero ó falso (1 ó 0).
    
    \BRterm{gls:cadena}{Cadena:} Es el \cdtRef{gls:tipoDato}{tipo de dato} definido por cualquier valor que se compone de una secuencia de caracteres, con o sin acentos, espacios, dígitos y signos de puntuación. Existen tres tipos de cadenas: palabra, frase y párrafo.
    
    \BRterm{gls:catalogo}{Catálogo:} Es una lista ordenada o clasificada de elementos relacionados.
    
    \BRterm{gls:decimal}{Decimal:} Es un \cdtRef{gls:tipoDato}{tipo de dato} \cdtRef{gls:numerico}{numérico}. Los números decimales son valores que denotan números racionales y la aproximación a números irracionales.
    
    \BRterm{gls:entero}{Entero:} Es el \cdtRef{gls:tipoDato}{tipo de dato} \cdtRef{gls:numerico}{numérico} definido por todos los valores numéricos enteros, tanto positivos como negativos.

    \BRterm{gls:entidad}{Entidad:} Término genérico que se utiliza para determinar un ente el cual puede ser concreto, abstracto o conceptual por ejemplo: Unidad administrativa, entregable, persona, etc. La entidades se caracterizan a través de atributos que personalizan a la entidad.		
    %Se usa para hacer referencia a un objeto con existencia física (entidad concreta) como: Una persona, un animal, una casa, etc.; o un objeto con existencia conceptual (entidad abstracta) como: Un puesto de trabajo, una asignatura de clases, un nombre, etc. Una \cdtRef{gls:entidad}{entidad} se representa por sus características o atributos, por ejemplo: La entidad persona tiene características como: Nombre, apellido, género, estatura, peso, fecha de nacimiento, etc.

    \BRterm{gls:fecha}{Fecha:} Es un \cdtRef{gls:tipoDato}{tipo de dato} que indica un día único en referencia al calendario gregoriano. Los tipos de fecha utilizados son: \cdtRef{gls:fechaCorta}{fecha corta} y \cdtRef{gls:fechaLarga}{fecha larga}. %con formato DD/MM/YYYY, por ejemplo: 24/02/2013.

    \BRterm{gls:fechaCorta}{Fecha corta:} Es la representación del \cdtRef{gls:tipoDato}{tipo de dato} \cdtRef{gls:fecha}{fecha} en la forma DD/MM/YYYY, por ejemplo: 24/02/2013.

    \BRterm{gls:fechaLarga}{Fecha larga:} Es la representación del \cdtRef{gls:tipoDato}{tipo de dato} \cdtRef{gls:fecha}{fecha} en la forma DD de MM del YYYY, por ejemplo: 24 de febrero del 2013.

    \BRterm{gls:frase}{Frase:} Es un \cdtRef{gls:tipoDato}{tipo de dato}  conformado por \cdtRef{gls:palabra}{palabras} y espacios.
    
    \BRterm{gls:numerico}{Numérico:} Es un \cdtRef{gls:tipoDato}{tipo de dato} que se compone de la combinación de los símbolos \textit{0,1,2,3,4,5,6,7,8,9,. y -.}  que expresan una cantidad en relación a su unidad.
    
    \BRterm{gls:opcional}{Opcional:} Es un elemento que el actor puede o no proporcionar en el formulario o la pantalla, su decisión no afectará la ejecución de la operación solicitada.

    \BRterm{gls:palabra}{Palabra:} Es un \cdtRef{gls:tipoDato}{tipo de dato} \cdtRef{gls:cadena}{cadena} conformado por el alfabeto y símbolos especiales como son \textit{\#,-,\$,\%,\&,(,),etc} y se caracteriza por no tener espacios.

    \BRterm{gls:parrafo}{Párrafo:} Es un \cdtRef{gls:tipoDato}{tipo de dato}  conformado por \cdtRef{gls:frase}{frases}.
    %\BRterm{gls:sn}{S/N} Abreviación del término ``Sin número'' utilizado para indicar cuando una Dirección Geográfica no tiene numeración.

    \BRterm{gls:requerido}{Requerido:} Es un \cdtRef{gls:tipoDato}{tipo de dato} que debe proporcionarse de manera obligatoria. La ejecución de la operación solicitada dependerá de que se proporcione este dato.
    
    %Es un atributo de una \cdtRef{gls:entidad}{entidad} que por definición no puede quedar indeterminado. Lo cual implica para el sistema, que, si se solicita mediante una pantalla, base de datos o servicio externo, el dato debe proporcionarse de manera obligatoria para el registro adecuado en el sistema.

    \BRterm{gls:tipoDato}{Tipo de dato:} Es el dominio o conjunto de valores que puede tomar un atributo de una \cdtRef{gls:entidad}{entidad} en el modelo de información. Los tipos de datos utilizados son: \cdtRef{gls:palabra}{palabra}, \cdtRef{gls:frase}{frase}, \cdtRef{gls:parrafo}{párrafo}, \cdtRef{gls:numerico}{numérico}, \cdtRef{gls:fecha}{fecha} y \cdtRef{gls:booleano}{booleano}.

    %\BRterm{gls:na}{NA} Abreviación del término ``No Aplica'', se utiliza para indicar que algún elemento en la estructura del documento o en el sistema no aplica.
\end{description}


%====================================================================
\section{Términos del negocio}
\label{gls:terminosNegocio}
En esta sección se definen los términos del negocio que se utilizan para comprender el comportamiento del sistema.

\begin{description}
    
    \BRterm{gls:TipoAuto}{Tipo de Automóvil:} Clase a la que pertenece el automóvil. Se utiliza como \cdtRef{gls:tipoDato}{tipo de dato} para el sistema, puede tomar alguno de los siguientes valores: ``Propio'' u ``Otro''.
    \BRterm{gls:sexo}{Sexo:} Sexo al que pertenece la persona. Se utiliza como \cdtRef{gls:tipoDato}{tipo de dato} para el sistema, puede tomar alguno de los siguientes valores: ``Masculino'' o ``Femenino''.
 	\BRterm{gls:TipoContacto}{Tipo de Contacto:} Clase a la que pertenece el contacto. Se utiliza como \cdtRef{gls:tipoDato}{tipo de dato} para el sistema, puede tomar alguno de los siguientes valores: ``Familia'', ``Emergencia'', ``Aseguradora''.
 	\BRterm{gls:TipNotif}{Tipo de Notificación:} Clase a la que pertenece la notificación. Se utiliza como \cdtRef{gls:tipoDato}{tipo de dato} para el sistema, puede tomar alguno de los siguientes valores: ``Manual'', ``Automatica''.
 	\BRterm{gls:TipoSangre}{Tipo de Sangre:} Grupo sanguíneo al que pertenece la sangre del usuario. Se utiliza como \cdtRef{gls:tipoDato}{tipo de dato} para el sistema, puede tomar alguno de los siguientes valores: ``$A+$'', `` $B+$ '', `` $O+$ '', `` $AB+$ '', `` $A-$ '', `` $B-$ '', `` $O-$ '', `` $AB-$ ''.
 	\BRterm{gls:TipoEnfermedad}{Tipo de Enfermedad:} Clase a la que pertenece la enfermedad. Se utiliza como \cdtRef{gls:tipoDato}{tipo de dato} para el sistema, puede tomar alguno de los siguientes valores: ``No Crónica'', ``Crónica''.
 	
 	


   
\end{description}
